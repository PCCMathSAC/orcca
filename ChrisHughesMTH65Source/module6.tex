%+*** mainfile.tex
% arara: makechapters: { files: [ mainfile], items: [module6] }
% !arara: indent: { overwrite: on, trace: yes, localSettings: on}
\chapter{Factoring}
\minitoc

\section{Factoring special products}
\textref{7.4}{438}%
We have so far considered factoring by grouping, reversing
the \gls{FOIL} procedure, and using the $ac$ method. In this section we will consider factoring
polynomials that have arisen from the following special products:
\begin{itemize}
	\item the sum and difference of two terms
	\item the square of the sum of two terms
	\item the square of the difference of two terms
\end{itemize} 

\subsection{The sum and difference of two terms}
Remember that when multiplying two binomials, one of which is the sum of two
terms and the other is the difference of the same the two terms, the FOILing procedure
gives a result that has no {\color{blue}{O}}utside or {\color{blue}{I}}nside terms. Consider the following examples
to refresh your memory
\[
	\begin{array}{lclcl}
		(x+2)(x-2)   & = & x^2-2x+2x-4         & = & x^2-4   \\
		(y-3)(y+3)   & = & y^2-3y+3y-9         & = & y^2-9   \\
		(2x+5)(2x-5) & = & 4x^2 -10x + 10x -25 & = & 4x^2-25 \\
		(4a-1)(4a+1) & = & 16a^2+4a-4a -1      & = & 16a^2-1 
	\end{array}
\]

We therefore see that the only terms we need to consider when FOILing these types of products are
the  {\color{blue}{F}}irst and  {\color{blue}{L}}ast terms. 

Note: the connector between the terms in the end result will always be a `-' sign if it has arisen
from the product of the sum and difference of two terms. If the connector is positive then
the \gls{polynomial} can not be factored (in the real numbers) and is called `\gls{prime}'.

These observations will help us in factoring expressions. The technique to factoring a difference of squares can
be described as follows
\begin{steps}
	\item Are there any common factors?
	\item \Gls{factor} the first term (find the square root of this term)
	\item Factor the last term (find the square root of this term)
	\item Create binomials with parentheses
	\item Put the first terms in each \gls{binomial}
	\item Put the last term in each binomial
	\item Put an addition sign between one binomial and a subtraction sign in between the other binomial
	\item As always, CHECK using the FOIL method
\end{steps} 
\label{list:stepsforfactoring}
\begin{myexample}\label{ex:factordiffsquares}
Factor
\[
	x^2-16
\]
{}
\end{myexample}
\begin{myProof}
	We follow the steps listed above
	\begin{steps}
		\item Are there any common factors?\hfill No
		\item	Factor the first term \hfill               $(x)^2$
		\item	Factor the last term  \hfill               $(4)^2$
		\item	Create the parenthesis\hfill               $(\phantom{x+4})(\phantom{x-4})$
		\item	Put the first term in each binomial\hfill  $(x\phantom{+4})(x\phantom{-4})$
		\item	Put the last terms in the binomial  \hfill $(x\phantom{+}4)(x\phantom{-}4)$
		\item	Put in one of each sign              \hfill $(x+4)(x-4)$
		\item	Check using FOIL                     \hfill exercise                     
	\end{steps}
	We conclude that 
	\[
		x^2-16 = (x+4)(x-4)
	\]
	{}
\end{myProof} 

\begin{myexample}
Factor
\[
	9x^2-25
\]
{}
\end{myexample}
\begin{myProof}
	We proceed as in \cref{ex:factordiffsquares}
	\begin{steps}
		\item	Are there any common factors?      \hfill  No                             
		\item	Factor the first term              \hfill  $(3x)^2$
		\item	Factor the last term               \hfill  $(5)^2$
		\item	Create the parenthesis             \hfill  $(\phantom{x+4})(\phantom{x-4}) $  
		\item	Put the first term in each binomial\hfill  $(3x\phantom{+4})(3x\phantom{-4})$ 
		\item	Put the last terms in the binomial \hfill  $(3x\phantom{+}5)(3x\phantom{-}5)$ 
		\item	Put in one of each sign            \hfill  $(3x+5)(3x-5)                    $ 
		\item	Check using FOIL                   \hfill  exercise                       
	\end{steps}
	We conclude that
	\[
		9x^2-25 = (3x+5)(3x-5)
	\]
	{}
\end{myProof} 

In some cases there might be a common factor which should be taken care of first. For example
\[
	2x^2-8 = 2(x^2-4)
\]
from which we can clearly move on as demonstrated in the previous examples. 

In other cases one of the binomials might need to be factored again. The next two examples 
show this.

\begin{myexample}
Factor
\[
	x^4-1
\]
\end{myexample}
\begin{myProof}
	Let's follow the steps we've followed in the previous examples:
	\begin{steps}
		\item	Are there any common factors?                          \hfill No                               
		\item	Factor the first term                                  \hfill $(x^2)^2$
		\item	Factor the last term                                   \hfill $(1)^2$
		\item	Create the parenthesis                                 \hfill $(\phantom{x+4})(\phantom{x-4})$     
		\item	Put the first term in each binomial                    \hfill $(x^2\phantom{+4})(x^2\phantom{-4})$
		\item	Put the last terms in the binomial                     \hfill $(x^2\phantom{+}1)(x^2\phantom{-}1)$ 
		\item	Put in one of each sign                                \hfill $(x^2+1)(x^2-1)$                     
		\item	ADDITIONAL STEP: Can any of the new terms be factored? \hfill Yes                              
		\item	Factor the second binomial                             \hfill $(x^2+1)(x-1)(x+1)$
		\item	Check using FOIL                                       \hfill exercise                         
	\end{steps}
	We conclude that
	\[
		x^4-1 = (x^2+1)(x+1)(x-1)
	\]
	{}
\end{myProof} 

\begin{myexample}
Factor
\[
	2x^3-8x
\]
{}
\end{myexample}
\begin{myProof}
	Off we go with the steps from \cpageref{list:stepsforfactoring}.
	\begin{steps}
		\item Are there any common factors?                \hfill Yes, $2x$    
		\item Factor out the common factor                 \hfill $2x(x^2-4)$    
		\item Does this contain the difference of squares? \hfill Yes        
		\item Factor using the previous techniques         \hfill $2x(x+2)(x-2)$
		\item Check                                        \hfill exercise   
	\end{steps}
	We conclude that
	\[
		2x^3-8x = 2x(x+2)(x-2)
	\]
	{}
\end{myProof} 

\begin{myexample}
\drillandskill
Factor the following 
\end{myexample}
\begin{multicols}{2}
	\begin{enumerate}
		\item $x^2-1$  \solution{$=(x-1)(x+1)$}
		\item $x^2-25$ \solution{$=(x-5)(x+5)$}
		\item $x^2-36$ \solution{$=(x-6)(x+6)$}
		\item $x^2-49$ \solution{$=(x-7)(x+7)$}
		\item $
		\begin{aligned}[t]
			x^4-1 & =  \solution{(x^2-1)(x^2+1)}    \\
			      & =  \solution{(x-1)(x+1)(x^2+1)} 
		\end{aligned}
		$
		\item $x^4-16$ \solution{$=(x-2)(x+2)(x^2+4)$}
		\item $x^4-81$ \solution{$=(x-3)(x+3)(x^2+9)$}
		\item $x^4-256$\solution{$=(x-4)(x+4)(x^2+16)$}
	\end{enumerate}
\end{multicols}

Factor the following- remove a common factor first!
\begin{multicols}{2}
	\begin{enumerate}
		\item $
		\begin{aligned}[t]
			x^3-x & =  \solution{x (x^2-1)}   \\
			      & =  \solution{x(x-1)(x+1)} 
		\end{aligned}$
		\item $x^3-25x$ \solution{$=x(x-5)(x+5)$}
		\item $x^3-36x$ \solution{$=x(x-6)(x+6)$}
		\item $x^3-49x$ \solution{$=x(x-7)(x+7)$}
	\end{enumerate}
\end{multicols}

Factor the following:
\begin{enumerate}
	\item $4x^2-1$        \solution{$=(2x-1)(2x+1)$}
	\item $9x^2-4$        \solution{$=(3x-2)(3x+2)$}
	\item $4x^2-25y^2$    \solution{$=(2x-5y)(2x+5y)$}
	\item $16a^4 - 36b^4$ \solution{$=4(2a^2-3b^2)(2a^2+3b^2)$}
\end{enumerate}

\subsection{The square of the sum or difference of two terms}
Remember from our previous work in FOILing polynomials, there are two special products
of binomials, which result in a `perfect square \gls{trinomial}'. This means that a binomial was
squared to get the trinomial. 

For example, we considered polynomials such as 
\begin{itemize}
	\item $(x+3)^2 = x^2+6x+9$
	\item $(3x-4)^2 = 9x^2-24x+16$
	\item $(5a+1)^2 = 25a^2+10a+1$
\end{itemize} 
Notice in using the FOIL method that the {\color{blue}O} and {\color{blue}I} 
terms are the same in each case, which means that the middle terms gets {\em doubled}. 
What we double is actually the product of the two terms. 

To determine if a trinomial is a perfect square trinomial, the following questions may be useful:
\begin{steps}
	\item Is the first term a perfect square? What from?	
	\item Is the last term positive and a perfect square? What form?
	\item Is the middle term double the product of the `what forms' above?
\end{steps} 
If we can answer each of these questions, then we have every chance of factoring a trinomial
that has arisen from squaring the sum or difference of two terms. We will demonstrate the technique
with examples.

\begin{myexample}
Factor
\[
	x^2+10x+25
\]
{}
\end{myexample}
\begin{myProof}
	\begin{steps}
		\item Is the first term a perfect square?          \hfill Yes.  What form? $x^2$
		\item Is the last term added and a perfect square? \hfill Yes.  What form? $5^2$ 
		\item Is the middle term double the product? \hfill $2(x)(5)=10x$. Yes! 
		\item Now rewrite as a perfect square binomial     \hfill $(x+5)^2$ 
		\item Check:
		$\begin{aligned}[t]
			(x+5)^2 & =		x^2+5x+5x+25 \\
			        & =		x^2+10x+25   
		\end{aligned}$
	\end{steps}
	{}
\end{myProof} 

\begin{myexample}
Factor
\[
	x^2-10x+25
\]
\end{myexample}
\begin{myProof}
	\begin{steps}
		\item Is the first term a perfect square?          \hfill Yes.    What form?  $x^2$
		\item Is the last term added and a perfect square? \hfill Yes.    What form?  $5^2$  or  $(-5)^2$
		\item Is the middle term double the product?        \hfill 
		$\begin{aligned}[t]
			2(x)(5)=10x     & \text{ No}  \\
			2(x)(-5) = -10x & \text{ Yes} 
		\end{aligned}$
		\item Now rewrite as a perfect square binomial     \hfill $(x-5)^2$
		\item Check:
		$\begin{aligned}[t]
			(x-5)^2 & =		x^2-5x-5x+25 \\
			        & =		x^2-10x+25   
		\end{aligned}$
	\end{steps}
	{}
\end{myProof} 

\begin{myexample}
Factor
\[
	18c^2+48c + 32
\]
\end{myexample}
\begin{myProof}
	\begin{steps}
		\item Is there a common factor?                    \hfill Yes, $2$ is the common factor 
		\item Factor this term out:                        \hfill $2(9c^2+24c+16)$
		\item Is the first term a perfect square?          \hfill Yes:  $3c$
		\item Is the last term added and a perfect square? \hfill Yes:  $4$                      
		\item Is the middle term double the product?       \hfill Yes:  $2(3c)(4)=24c$
		\item Rewrite as a binomial                        \hfill $2(3c+4)^2$                     
		\item Check: 
		$\begin{aligned}[t]
			2(3c+4)^2 & = 2(9c^2+24c+16) \\
			          & = 18c^2+48c+32   
		\end{aligned}$ 
	\end{steps}
	{}
\end{myProof} 

\section{Factoring sums and differences of cubes}
In the previous module we considered factoring special products; in particular, this 
included
\begin{itemize}
	\item the product of the sum and difference of two terms, e.g $(x-y)(x+y)=x^2-y^2$
	\item the square of the sum of two terms e.g $(x+y)^2=x^2+2xy+y^2$
	\item the square of the difference of two terms e.g $(x-y)^2=x^2-2xy+y^2$
\end{itemize} 
We begin this module by considering factoring the sum or difference of two cubes of the form
\[
	x^3+y^3, \qquad  x^3-y^3
\]

\subsection{Factoring the sum of two cubes}
We begin this discussion with an example in multiplying two polynomials- this will hopefully
make the process of factoring more transparent.

\begin{myexample}\label{ex:sumcubes}
Multiply 
\[
	(x+y)(x^2-xy+y^2)
\]
\end{myexample}
\begin{myProof}
	We use vertical format for this problem, and write like terms underneath one another:
	\begin{alignat*}{3}
		(x+y)(x^2-xy+y^2) & =  x^3 & -x^2y & +xy^2 &      \\
		                  &        & +yx^2 & -xy^2 & +y^3 \\
	\end{alignat*}
	Notice here that the two `mixed' terms consisting of both $x$ and $y$ cancel one another, 
	and so the result that we are left with is
	\[
		(x+y)(x^2-xy+y^2) = x^3+y^3
	\]  
	This will clearly be useful to us when factoring. We now demonstrate this technique with some examples.
\end{myProof} 

\begin{myexample}
Factor
\[
	x^3+8
\]
\end{myexample}
\begin{myProof}
	In this example it is crucial to note that $8=2^3$. This means that we can write our \gls{expression}
	as
	\[
		x^3 +8 = x^3 + 2^3
	\]
	which is the sum of two cubes, and means that the work we described in \cref{ex:sumcubes}
	will be very useful to us. 
				
	Therefore
	\begin{align*}
		x^3+2^3 & =  (x+2)(x^2-2x+2^2) \\
		        & =  (x+2)(x^2-2x+4)   
	\end{align*} 
	As always, it is vital to check that this works by multiplying the result (exercise).
\end{myProof} 

\begin{myexample}\label{ex:anothersumcubes}
Factor
\[
	8x^3+27
\]
\end{myexample}
\begin{myProof}
	In this example we first need to realize that, by the properties of exponents, $8x^3 = (2x)^3$, and
	that $27=3^3$. For our examples, there will usually be a `nice' factorization. We can use this
	information and put it together with our theory so far
	\begin{align*}
		8x^3+27 & =  (2x)^3+3^3                 \\
		        & =  (2x+3)((2x)^2-(2x)(3)+3^2) \\
		        & =  (2x+3)(4x^2-6x +9)         
	\end{align*}{} 
	As in the previous example, we must always check our answer (exercise).
\end{myProof} 

\begin{myexample}
\drillandskill
Factor the following:
\end{myexample}

\begin{multicols}{2}
	\begin{enumerate}
		\item $x^3+1$    \solution{$=(x+1)(x^2-x+1)$}
		\item $x^3+27$   \solution{$=(x+3)(x^2-3x+9)$}
		\item $x^3+64$   \solution{$=(x+4)(x^2-4x+16)$}
		\item $x^3+125$  \solution{$=(x+5)(x^2-5x+25)$}
		\item $8x^3+1$   \solution{$=(2x+1)(4x^2-2x+1)$}
		\item $8x^3+27$  \solution{$=(2x+3)(4x^2-6x+9)$}
		\item $8x^3+27y^3$\solution{$=(2x+3y)(4x^2-6xy+9y^2)$}
		\item $27x^6+64$  \solution{=$(3x^2+4)(9x^2-12x^2+16)$}
	\end{enumerate}
\end{multicols}

\subsection{Factoring the difference of two cubes}
This discussion parallels the technique used to factor the sum of two cubes. We begin by
demonstrating an example involving polynomial multiplication, and then move onto factorization.

\begin{myexample}
Multiply
\[
	(x-y)(x^2+xy+y^2)
\]
\end{myexample}
\begin{myProof}
	We use the vertical format
	\begin{alignat*}{3}
		(x-y)(x^2+xy+y^2) & =  x^3 & +x^2y & +x y^2 &      \\
		                  &        & -yx^2 & -xy^2  & -y^3 
	\end{alignat*}
	We notice that the mixed terms cancel, and that we are left with
	\[
		(x-y)(x^2+xy+y^3) = x^3-y^3
	\]
	We will use this when factoring the difference of two cubes.
\end{myProof} 

\begin{myexample}
Factor
\[
	y^3-64
\]
\end{myexample}
\begin{myProof}
	The crucial observation that we must make here is that $64=4^3$. This therefore means
	that
	\begin{align*}
		y^3-64 & =  y^3 - 4^3             \\
		       & =  (y-4)(y^2+(y)(4)+4^2) \\
		       & =  (y-4)(y^2+4y+16)      
	\end{align*}
	{}
\end{myProof} 

\begin{myexample}
Use \cref{ex:anothersumcubes} to help you factor
\[
	8x^3-27
\]
\end{myexample}
\begin{myProof}
	As we observed in \cref{ex:anothersumcubes}, $8x^3 = (2x)^3$, and $27=3^3$. This allows
	us to write
	\begin{align*}
		8x^3-27 & =  (2x)^3 - 3^3                 \\
		        & =  (2x-3)((2x)^2 + (2x)(3)+3^2) \\
		        & =  (2x-3)(4x^2 + 6x + 9)        
	\end{align*} 
\end{myProof} 

\begin{myexample}
\drillandskill
Factor the following:
\begin{multicols}{2}
	\begin{enumerate}
		\item $x^3-1$       \solution{$=(x-1)(x^2+x+1)$}
		\item $x^3-27$      \solution{$=(x-3)(x^2+3x+9)$}
		\item $x^3-64$      \solution{$=(x-4)(x^2+4x+16)$}
		\item $x^3-125$     \solution{$=(x-5)(x^2+5x+25)$}
		\item $a^3-b^3$     \solution{$=(a-b)(a^2+ab+b^2)$}
		\item $27a^3-b^3$   \solution{$=(3a-b)(9a^2+3ab+b^2)$}
		\item $64a^3-b^3$   \solution{$=(4a-b)(16a^2+4ab+b^2)$}
		\item $64a^3-b^3c^3$\solution{$=(4a-bc)(16a^2+4abc+b^2c^2)$}
	\end{enumerate}
\end{multicols}

\end{myexample}

\begin{myexample}
\drillandskill
Factor the following miscellaneous expressions:
\end{myexample}
\begin{multicols}{2}
	\begin{enumerate}
		\item $x^4+x$    \solution{$=x(x+1)(x^2-x+1)$}
		\item $x^4+27x$  \solution{$=x(x+3)(x^2-3x+9)$}
		\item $x^4+64x$  \solution{$=x(x+4)(x^2-4x+16)$}
		\item $x^4+125x$ \solution{$=x(x+5)(x^2-5x+25)$}
		\item $x^4-x$    \solution{$=x(x-1)(x^2+x+1)$}
		\item $x^4-27x$  \solution{$=x(x-3)(x^2+3x+9)$}
		\item $x^4-64x$  \solution{$=x(x-4)(x^2+4x+16)$}
		\item $x^4-125x$ \solution{$=x(x-5)(x^2+5x+25)$}
	\end{enumerate}
\end{multicols}

\section{Solving quadratic equations by factoring}
\textref{7.6}{454}%
\reformatstepslist{Q} % the steps list should be P1, P2, \ldots
In this class we have so far considered factoring polynomials and trinomials in a number of different
ways. One of the many applications of this skill is in solving {\em \gls{quadratic} equations}. A quadratic
\gls{equation} has the form
\[
	ax^2+bx+c=0
\]
where $a$, $b$, and $c$ are real numbers, and we assume that $a\ne 0$. We have actually been working
with quadratic expressions a lot so far in our work, but we have been calling them 2nd \gls{degree} polynomials, or
trinomials. Each description is equivalent.

In this section we will discuss the zero product rule, and demonstrate how to use the zero product rule together
with factorization in order to \gls{solve} quadratic equations. 

\begin{myDefinition}\label{sec:zeroprodprin}
	%\subsection{Zero product rule}\label{sec:zeroprodprin}
	{Zero product principle}: 
	This rule says that if the product of two numbers is 0, then either the first one equals 0, or the second one
	equals 0, or they both equal 0. 
				
	Mathematically, we say that if
	\[
		AB=0
	\]
	then either $A=0$ or $B=0$.
\end{myDefinition}

This property will allow us to solve quadratic equations by first factoring the quadratic expression, then set
each of the factors equal to 0, then solve each equation.

We demonstrate this with an example.

\begin{myexample}
Solve the quadratic equation
\[
	x^2+3x+2=0
\]
\end{myexample}
\begin{myProof}
	If we are to use the zero product principle, then we must first factor the quadratic. Remember that we 
	are looking for factors of 2 that add up to 3. We therefore see that this equation can be written as
	\[
		(x+2)(x+1)=0
	\]
	The zero product rule tells us that either
	\[
		x+2=0\qquad or \qquad x+1 = 0
	\]
	Solving these two equations should pose no problem to us, and we see that these equations give
	\[
		x=-2 \qquad or \qquad x=-1
	\]
	It remains to check our answers, which we do by plugging them back into the original equation
	\[
		(-2)^2 + 3(-2)+2 = 0 \qquad {\mathrm and}\qquad (-1)^2+3(-1)+2=0
	\]
	{}
\end{myProof} 

This method relies on being able to factor. This may take time to develop so be patient. We can
summarize the procedure for solving a quadratic equation by factoring as follows
\begin{steps}
	\item Put it in standard form with all terms involving $x$ on one side of the = sign
	\item Factor our any common factors
	\item Factor what is left
	\item Set each factor with a \gls{variable} in it to 0
	\item Solve each individual equation
	\item Place all answers in set notation
	\item Check in the original equation to see if they do indeed make the equation true
\end{steps} 

If there are no solutions to the given equation then the overall \gls{solution} is the empty set which
can be written as either $\{\}$ or $\emptyset$. In the event that every number will work for this equation
then the solution is All Real Numbers.

\begin{myexample}
Use factoring to solve the quadratic equation
\[
	x^2+x-42 = 0
\]
\end{myexample}
\begin{myProof}
	\begin{steps}
		\item Is it in standard form and equal to 0?     \hfill Yes               
		\item	Are there any common factors?              \hfill No                
		\item	Factor                                     \hfill $(x+7)(x-6)=0$        
		\item	Set each factor with a variable equal to 0 \hfill $x+7=0$  and  $x-6=0$ 
		\item	Solve each equation                        \hfill $x=-7$  and  $x=6$    
		\item	Put your answers in set notation           \hfill $\{-7,6\}$
		\item	Finally we check our answers
		\[
			(-7)^2+(-7)-42 = 0 \qquad (6)^2+6-42 = 0
		\]
		both of which are true. We conclude that the solutions to the given equation are
		\[
			\{-7,6\}
		\]
	\end{steps}
	{}
\end{myProof}

The next example is different from most that we have considered previously, as it is not a trinomial
(a polynomial with three terms). The process that we use to solve this type of equation follows the 
same methodology, as we will demonstrate.

\begin{myexample}
Solve the quadratic equation by factoring
\[
	x^2=8x
\]
\end{myexample}
\begin{myProof}
	\begin{steps}
		\item Is it in standard form and equal to 0? \hfill No  $x^2-8x=0$
		\item	Are there any common factors?          \hfill No              
		\item	Factor                                 \hfill $x(x-8)=0$          
		\item	Set up equations:                      \hfill $x=0$  and  $x-8=0$ 
		\item	Solve                                  \hfill $x=0$  and  $x=8$   
		\item	Put answers in set notation            \hfill $\{0,8\}$           
		\item	Finally we check our answers
		\[
			0^2 = 8(0) \qquad 8^2 = 8(8)
		\]
		both of which are true, and we conclude that the solutions to the given equation are
		\[
			\{0,8\}
		\]
	\end{steps}
	{}
\end{myProof} 

In our examples so far there have been two distinct (unequal) answers. This may not always be the case, 
as we will see in the next example. 

\begin{myexample}
Solve the quadratic equation by factoring
\[
	9x^2-30x = -25
\]
\end{myexample}
\begin{myProof}
	\begin{steps}
		\item Put in standard form and equal to 0 \hfill $9x^2-30x+25 = 0$                     
		\item	Are there any common factors?       \hfill No                                
		\item	Factor                              \hfill $(3x-5)(3x-5)=0$                      
		\item	Set up equations:                   \hfill $3x-5=0$  and  $3x-5=0$               
		\item	Solve                               \hfill $x=\frac{5}{3}$  and  $x=\frac{5}{3}$ 
		\item	Put answers in set notation         \hfill $\left\{\frac{5}{3}\right\}$
		\item Check \hfill exercise
	\end{steps}
	Note that there is no need to write the the same number twice. After checking (exercise), we conclude that
	the solution set to the given equation is
	\[
		\left\{\frac{5}{3}\right\}
	\]
	It seems like there is only one answer here, but in fact we say that there are two and that the solution is
	repeated. 
\end{myProof}

We have so far seen examples that demonstrate that it is possible to have {\em two} real solutions, or {\em one} real
solution. The only other case is when there are {\em no} real solutions; we will study this case in a later module. For
the moment we continue with some further examples. 

\begin{myexample}
Solve the following quadratic equation by factoring
\[
	25x^2=49
\]
\end{myexample}
\begin{myProof}
	\begin{steps}
		\item Put in standard form and equal to 0 \hfill $25x^2-49=0$                           
		\item	Are there any common factors?       \hfill No                                 
		\item	Factor                              \hfill $(5x-7)(5x+7)=0$                       
		\item	Set up equations:                   \hfill $5x-7=0$  and  $5x+7=0$                
		\item	Solve                               \hfill $x=\frac{7}{5}$  and  $x=-\frac{7}{5}$ 
		\item	Put answers in set notation         \hfill $\left\{\frac{7}{5}, -\frac{7}{5}\right\}$
		\item Check   \hfill exercise
	\end{steps}
	Once we have checked (exercise), we can conclude that the solution to the given equation is
	\[
		\left\{\frac{7}{5}, -\frac{7}{5}\right\}
	\] 
	When we have two solutions that are the same but with opposite sign, we can use the following abbreviation
	\[
		x = \pm \frac{7}{5}
	\]
	We read this as `$x$ equals plus or minus 7 over 5'.
\end{myProof} 

\begin{myexample}
Solve the following equations by factoring.
\drillandskill
\end{myexample}

{\em Straight forward}
\begin{multicols}{2}
	\begin{enumerate}
		\item $x^2+5x+4=0$  \solution{$\left\{-1,-4\right\}$}
		\item $x^2+14x+13=0$ \solution{$\left\{-1,-13\right\}$}
		\item $x^2-4x-5=0$ \solution{$\left\{5,-1\right\}$}
		\item $x^2-5x-6=0$ \solution{$\left\{6,-1\right\}$}
		\item $x^2+12x-64=0$ \solution{$\left\{4,-16\right\}$}
		\item $x^2-22x+72=0$ \solution{$\left\{18,4\right\}$}
		\item $x^2-x-30=0$ \solution{$\left\{6,-5\right\}$}
		\item $x^2-2x-35=0$ \solution{$\left\{7,-5\right\}$}
	\end{enumerate}
\end{multicols}
{\em Slightly harder}
\begin{multicols}{2}
	\begin{enumerate}
		\item $x^2+5x=-4$  \solution{$\left\{-1,-4\right\}$}
		\item $x^2-18x=-80$ \solution{$\left\{10,8\right\}$}
		\item $x^2=-6x-8$ \solution{$\left\{-2,-4\right\}$}
		\item $x^2=3x+18$ \solution{$\left\{6,-3\right\}$}
		\item $10x-48 = -2x^2$ \solution{$\left\{3,-8\right\}$}
		\item $2x^2+24 = 14x$ \solution{$\left\{4,3\right\}$}
		\item $-3x-10 = -x^2$ \solution{$\left\{5,-2\right\}$}
		\item $45x = -15x^2+60$ \solution{$\left\{1,-4\right\}$}
	\end{enumerate}
\end{multicols}

{$a\ne 1$}
\begin{multicols}{2}
	\begin{enumerate}
		\item $2x^2+3x+1=0$ \solution{$\left\{-\frac{1}{2}, -1\right\}$}
		\item $2x^2+19x+35=0$ \solution{$\left\{-\frac{5}{2}, -7\right\}$}
		\item $3x^2+14x-5=0$ \solution{$\left\{\frac{1}{3}, -5\right\}$}
		\item $6w^2-11w+4=0$ \solution{$\left\{\frac{4}{3}, \frac{1}{2}\right\}$}
	\end{enumerate}
\end{multicols}

{\em Difference of squares}
\begin{multicols}{2}
	\begin{enumerate}
		\item $x^2-25=0$ \solution{$\left\{\pm 5\right\}$}
		\item $x^2 = 4$ \solution{$\left\{\pm 2 \right\}$}
		\item $x^2 = 16$ \solution{$\left\{\pm 4\right\}$}
		\item $x^2 = 64$ \solution{$\left\{\pm 8\right\}$}
		\item $4x^2 = 16$ \solution{$\left\{\pm 2\right\}$}
		\item $-x^2 = -9$ \solution{$\left\{\pm 3\right\}$}
		\item $3x^2 = 75$ \solution{$\left\{\pm 5\right\}$}
		\item $81x^2 = 25$ \solution{$\left\{\pm \frac{5}{9}\right\}$}
	\end{enumerate}
\end{multicols}

{$(a+b)^2$ or $(a-b)^2$}
\begin{multicols}{2}
	\begin{enumerate}
		\item $x^2+4x+4 = 0$ \solution{$\left\{-2\right\}$}
		\item $x^2-4x+4 = 0$ \solution{$\left\{2\right\}$}
		\item $x^2+8x+16 = 0$ \solution{$\left\{-4\right\}$}
		\item $x^2+16 = 8x$ \solution{$\left\{4\right\}$}
		\item $2x^2+20x+50 = 0$ \solution{$\left\{-5\right\}$}
		\item $3x^2+12x+12=0$ \solution{$\left\{-2\right\}$}
	\end{enumerate}
\end{multicols}

\begin{multicols}{2}
	\begin{enumerate}
		\item $x^2+2x = -4x-8$ \solution{$\left\{-2,-4\right\}$}
		\item $x^2+3x = -5x-15$ \solution{$\left\{-3,-5\right\}$}
		\item $x^2+7x = 4x+28$ \solution{$\left\{4,-7\right\}$}
		\item $x^2+3x = 5x+15$ \solution{$\left\{5,-3\right\}$}
	\end{enumerate}
\end{multicols}


\section{Solving quadratic equations and problem solving}\label{sec:solvequadfac}
In the previous section we introduced quadratic equations, and they have the form
\[
	ax^2+bx+c=0
\]
where $a$, $b$, and $c$ are real numbers and we assume that $a\ne 0$. We learnt how we
can solve quadratic equations (i.e find the values of $x$ that satisfy the equation), by
factoring the equation and using the {\em Zero product principle} (see 
\cpageref{sec:zeroprodprin}). In this section we will apply 
our factoring skills and knowledge of quadratic equations to some word problems. We will work through
a series of examples to demonstrate approaches and techniques.

\begin{myexample}
The area of a square is 121 square units. Find the length of its sides.
\begin{center}
	\begin{tikzpicture}
		\draw (0,0)--(2,0)--(2,2)--(0,2)--cycle;
		\draw (-.5,1) node {$x$};
		\draw (1,-.5) node {$x$};
	\end{tikzpicture}
\end{center}

\end{myexample}
\begin{myProof}
	\begin{itemize}
		\item The first key piece of information that we need to remember about squares
		is that they have sides of equal length. This means that if we let one of
		the lengths be $x$, then the other sides must also have length $x$, as shown in the
		above diagram.
		\item The second piece of information that we need to remember about squares is that the area is
		the (length of one side)$^2$. In our example, this therefore translates to
		\[
			{\mathrm{area~of~square} = x^2}
		\]
		\item Combining this information with the fact that the area of this square is 121 square units gives
		the equation
		\[
			x^2=121
		\]	
	\end{itemize} 
	Remember from our work with factoring (see \cref{sec:solvequadfac}) that the first step is to write the equation in standard form, in
	other words with all of the terms on one side
	\[
		x^2-121=0
	\]
	On remembering that $121=11^2$, we realize that this is one of the special products discussed previously, so we can write it as
	\[
		(x-11)(x+11)=0
	\]
	We can now use the principle of zero products which means that we have to solve the two equations
	\[
		x-11=0, \qquad x+11=0
	\]
	At this stage we have to think back to what $x$ represents in this problem. Recall that $x$ is the length
	of one of the sides of the square, so the questions we must ask ourselves are:
	\begin{itemize}
		\item can a square have two different values for the length of its sides?
		\item can a square have a negative value for one of its lengths?
	\end{itemize}
	The answer to both of these questions is clearly {\em no}, so at this stage we must clearly reject the solution $x=-11$ as it is unrealistic.
				
	We therefore conclude that the length of side of a square that has an area of 121 square units is 11 units. 
				
	We check our answer by calculating $11^2=121$.
	{}
\end{myProof}

\begin{myexample}
An object is thrown upward from the top of an 80 foot building with an initial
velocity of 64 feet per second. The height of the object above the ground, $h$, after
$t$ seconds is given by
\[
	h = -16t^2+64t+80
\]
When will the object hit the ground? 
\end{myexample}
\begin{myProof}
	We have considered an example similar to this in \vref{ex:fallingobject}. In the current context
	we are concerned with the time when the object will hit the ground; since $h$ is the height of the {\em object
		above the ground}, when the object is on the ground this corresponds to $h=0$. So, we need to solve the equation
	\[
		0=-16t^2+64t+80
	\]
	Our approach will be
	\begin{itemize}
		\item factor the equation
		\item use the zero product rule to find the values of $t$ that satisfy the equations
		\item check our answer
	\end{itemize} 
				
	We begin by factoring the equation. At first glance, it seems intimidating since the coefficients are all fairly
	large numbers. However, note first that we can remove a factor of $16$ from each of the terms, so we can therefore
	write the equation as
	\[
		0=-t^2+4t+5
	\]
	Note: It is certainly possible to continue factoring at this stage, but the factor of $-1$ in front of the $t^2$ complicates
	matters somewhat; we can remove it by multiplying both sides of the equation by $-1$ to give
	\[
		0 = t^2-4t-5
	\]
	This equation is now significantly more friendly than the original, and using the techniques described in previous modules we see
	that we can factor is as
	\[
		0=(t-5)(t+1)
	\]
	and therefore by the zero product rule,
	\[
		t=5 {\mathrm{~or~}} t=-1
	\]
	In principal we have two values of $t$, but there are two questions that we must ask ourselves
	\begin{itemize}
		\item {\color{red}can there be two values of $t$ in this problem?} Theoretically speaking (in general), yes there could be. If the object
		were starting at a height of $0ft$ above the ground then $t=-1$ could represent this. However, we know that the object is starting
		at a height of 80ft above the ground, son in this particular example, the answer is NO.
		\item {\color{red}can $t$ be negative?} In this example $t$ represents time. For the duration of our class we will assume that $t$ can not 
		be negative, so the answer to this question is NO.
	\end{itemize} 
				
	We therefore conclude that $t=5$ is the time at which the object hits the ground. We must check our answer
	\[
		-16(5)^2+64(5)+80=0
	\]
	We can confirm this graphically by plotting values of $h$ against values of $t$ as shown below in 
	\cref{fig:thrownobject}
				
	\begin{figure}[!h]
		\centering
		\begin{tikzpicture}
			\begin{axis}[
					framed,
					xmin=-1,xmax=6,
					ymin=-20,ymax=150,
					xlabel={$t$},
					ylabel={$h$},
					xtick={0,...,5},
					ytick={0,20,...,200},
					grid=major
				]
				\addplot+[-]expression[domain=0:5,samples=100]{-16*x^2+64*x+80};
				\legend{$h(t)=-16t^t+64t+80$};
			\end{axis}
		\end{tikzpicture}
		\caption{A thrown object}
		\label{fig:thrownobject}
	\end{figure}
	\FloatBarrier
				
	The \gls{point} where the graph cuts the horizontal axis at $t=5$ is called the {\em horizontal \gls{intercept}} and in our example
	represents the value of $t$ at which the height of the object is $0$. 
				
	In general quadratics will have {\em at most} two horizontal intercepts.
\end{myProof}
