%+*** mainfile.tex
% arara: pdflatex: { files: [ mainfile.tex ] }
% !arara: indent: { overwrite: on, trace: yes, localSettings: on}
\chapter{Radicals}
\minitoc
\section{Finding roots}
\textref{9.1}{558}%
In the previous module we considered factoring \gls{quadratic} {\em expressions} of the form
\[
	ax^2+bx+c
\]
and applied the skill to solving quadratic {\em equations} of the form
\[
	ax^2+bx+c=0
\]
This process works well for equations that can be factored, but we will soon encounter
equations that can not be factored simply. In this case, we will need another tool to help
us, and in order to develop this tool we need to discuss {\em square roots}. In fact
we will also briefly discuss cubic roots. We begin with some terminology.

\begin{myDefinition}
	Terminology:
	\begin{itemize}
		\item Square roots- if $b^2=a$ then $b$ is a square root of $a$
		\item Radical sign- this is the symbol to indicate that a root is required $\sqrt{}$
		\item Radicand - the terms inside the radical sign, for example in $\sqrt{9}$ the radicand is $9$
	\end{itemize} 
	{}
\end{myDefinition} 

\begin{myexample}
Find the following
\begin{multicols}{3}
	\begin{enumerate}
		\item $\sqrt{9}$
		\item $\sqrt{25}$
		\item $\sqrt{64}$
	\end{enumerate} 
\end{multicols}
\end{myexample}
\begin{myProof}
	\begin{enumerate}
		\item We are looking for a number that when squared gives us $9$. We see that $3^2=9$, and also that
		$(-3)^2=9$. The {\em principal square root} of $9$ is 3; if we wanted the negative square root, then
		we would write $-\sqrt(9)$. We therefore conclude that
		\[
			\sqrt{9}=3
		\]
		\item Using a similar argument to part a
		\[
			\sqrt{25} = 5
		\]
		\item And finally
		\[
			\sqrt{64} = 8
		\]
	\end{enumerate} 
\end{myProof} 

You might have examined finding square roots of numbers in previous classes. Perhaps you
are used to the symbol but not necessarily the terminology. The numbers we encountered
in this example worked out very nicely, in that the results were whole numbers (integers). When
a number results from squaring an integer, we call it a {\em perfect square}. 

\begin{myDefinition}
	Principle square root: If $a$ is a non-negative real number, the non-negative number $b$ such that $b^2=a$, denoted
	by $b=\sqrt{a}$, is the principal square root of $a$. 
				
	Perfect squares: When the square root of a number is an integer, that number is a perfect square. For example, 
	100 is a perfect square since $\sqrt{100}=10$, and $17$ is not a perfect square.
\end{myDefinition}

Note: for the purposes of this class we will only consider taking the square root of non-negative numbers; if 
ever we do find ourselves in the situation of taking the square root of a negative number, we will give the answer
as `not a real number'.

\begin{myexample}
Find the following
\begin{multicols}{3}
	\begin{enumerate}
		\item $\sqrt{36}$
		\item $-\sqrt{49}$
		\item $-\sqrt{81}$
	\end{enumerate} 
\end{multicols}
\end{myexample}
\begin{myProof}
	\begin{enumerate}
		\item $\sqrt{36}=6$
		\item $-\sqrt{49}=-7$
		\item $-\sqrt{81}=-9$
	\end{enumerate} 
\end{myProof}
Note that the - sign in front of the radical symbol indicates taking the negative square root of the number,

We have so far discussed square roots, and in fact there are an infinity of other roots that we can consider. For example
\begin{myDefinition}
	Cube root: If $b^3=a$ then $b$ is the cube root of $a$, and we write $b=\sqrt[3]{a}$
					
	Fourth root: If $b^4=a$ then $b$ is a fourth root of $a$, and we write $b=\sqrt[4]{a}$
\end{myDefinition}


Notice the subtle use of language here; for the cube root we said that $b$ is {\em the} cube root of $a$, where
as for the fourth root, we said that $b$ is  {\em a } fourth root of $a$.

The {\em index} of radical is the number found in the crook of the radical sign, and indicates which root is
desired. Although the index is not written for finding square roots, it is in fact understood to 
be 2. Technically the square root could look like $\sqrt[2]{}$, but since square roots are used so frequently, 
the index is dropped. 

If the index is even, the radicand must be non-negative for the root to be a real number.

\begin{myexample}
Find the following $n^{th}$ roots; if they do not exist as a real number then say so.
\begin{multicols}{4}
	\begin{enumerate}
		\item $\sqrt[3]{-27}$
		\item $\sqrt[4]{81}$
		\item $\sqrt[4]{-8}$
		\item $\sqrt[3]{8}$
	\end{enumerate}
\end{multicols}
{}
\end{myexample}
\begin{myProof}
	\begin{enumerate}
		\item $\sqrt[3]{-27} = -3$ because $(-3)^3 = -27$. Notice how we {\em do} need parenthesis here
		\item $\sqrt[4]{81} = 3$ because $3^4 = 81$. Notice how we do {\em not} need parenthesis here.
		\item $\sqrt[4]{-8}$ does not exist as a real number because there is no real value of $x$ such that $x^4 = -81$. In fact, provided the
		index of the radical symbol is even, we necessarily require the radicand to be non-negative.
		\item $\sqrt[3]{8} = 2$ because $2^3 = 8$.
	\end{enumerate}
	{}
\end{myProof}

\begin{myexample}
\drillandskill
\Gls{simplify} the following
\end{myexample}
{\em Exponent reminder: Evaluate the following}:
\begin{multicols}{4}
	\begin{enumerate}
		\item $3^2$  \solution{$=9$}
		\item $(-3)^2$ \solution{$=9$}
		\item $-3^2$ \solution{$=-9$}
		\item $4^2$ \solution{$=16$}
		\item $5^2$ \solution{$=25$}
		\item $6^2$ \solution{$=36$}
		\item $7^2$ \solution{$=49$}
		\item $8^2$ \solution{$=64$}
		\item $2^3$ \solution{$=8$}
		\item $3^3$ \solution{$=27$}
		\item $4^3$ \solution{$=64$}
		\item $5^3$ \solution{$=125$}
		\item $(-2)^3$ \solution{$=-8$}
		\item $(-3)^3$ \solution{$=-27$}
		\item $(-4)^3$ \solution{$=-64$}
		\item $(-5)^3$ \solution{$=-125$}
		\item $1^4$ \solution{$=1$}
		\item $2^4$ \solution{$=16$}
		\item $3^4$ \solution{$=81$}
		\item $4^4$ \solution{$=256$}
	\end{enumerate}
\end{multicols}
{\em Find the following square roots}
\begin{multicols}{4}
	\begin{enumerate}
		\item $\sqrt{4}$ \solution{$=2$}
		\item $\sqrt{9}$ \solution{$=3$}
		\item $\sqrt{25}$ \solution{$=5$}
		\item $-\sqrt{16}$ \solution{$=-4$}
		\item $-\sqrt{36}$ \solution{$=-6$}
		\item $\sqrt{49}$ \solution{$=7$}
		\item $\sqrt{100}$ \solution{$=10$}
		\item $\sqrt{144}$ \solution{$=12$}
	\end{enumerate}
\end{multicols}

\section{Simplifying radical expressions}
\textref{9.2}{558}%
In the previous section we introduced radicals. In this section we will see examples of how to manipulate and
simplify some elementary radical expressions (this will be built upon significantly in Math 95). 

\subsection{Multiplying radical expressions}
\begin{myexample}
Find the following 
\begin{multicols}{2}
	\begin{enumerate}
		\item $\sqrt{9}\sqrt{16}$
		\item $\sqrt{9}\sqrt{25}$
	\end{enumerate} 
\end{multicols}
\end{myexample}
\begin{myProof}
	\begin{enumerate}
		\item 
		$\begin{aligned}[t]
			\sqrt{9}\sqrt{16} & =  3(4) \\
			                  & = 12    
		\end{aligned}$
		\item 
		$\begin{aligned}[t]
			\sqrt{4}\sqrt{25} & =  2(5) \\
			                  & = 10    
		\end{aligned}$
	\end{enumerate} 
\end{myProof} 
Now let's try some experimentation with some related examples. Consider
\begin{align*}
	\sqrt{9(16)} & =		\sqrt{144} \\
	             & =		12         
\end{align*} 
Compare this to part a). What do you notice? Next consider
\begin{align*}
	\sqrt{4(25)} & =		\sqrt{100} \\
	             & =		10         
\end{align*} 
Compare this to part b). What do you notice?

Hopefully you can see that
\[
	\sqrt{9(16)} = \sqrt{9}\cdot \sqrt{16}
\]
and that
\[
	\sqrt{4(25)} = \sqrt{4}\cdot\sqrt{25}
\]

In other words, when multiplying inside the radical sign we can distribute the radical
symbol to each \gls{factor}. We can use this in simplifying radical expressions that are not
perfect squares.

\begin{myexample}\label{ex:multradicals}
Simplify the following
\begin{multicols}{3}
	\begin{enumerate}
		\item $\sqrt{50}$
		\item $\sqrt{90}$
		\item $\sqrt{2x^6}$
	\end{enumerate} 
\end{multicols}
\end{myexample}
\begin{myProof}
	\begin{enumerate}
		\item 
		$\begin{aligned}[t]
			\sqrt{50} & =  \sqrt{2(25)}      \\
			          & =  \sqrt{2}\sqrt{25} \\
			          & =  5\sqrt{2}         
		\end{aligned}$
						
		Note: the reason we chose to write 50 as $50=2(25)$ is that 25 is a perfect square.
		\item 
		$\begin{aligned}[t]
			\sqrt{90} & =  \sqrt{9(10)}      \\
			          & =  \sqrt{9}\sqrt{10} \\
			          & =  3\sqrt{10}        
		\end{aligned}$
						
		Note:  we chose to write 90 as $90=9(10)$ as 9 is a perfect square. We can not simplify this any
		further as 10 can not be factored into a product involving perfect squares.
		\item 
		$\begin{aligned}[t]
			\sqrt{2x^6} & =  \sqrt{2}\sqrt{x^6} \\
			            & =  \sqrt{2}x^3        
		\end{aligned}$
						
		Note: the reason that $\sqrt{x^6}=x^3$is that $(x^3)^2 = x^6$ (by the properties of exponents).
	\end{enumerate} 
\end{myProof} 

\begin{myDefinition}
	In the above examples we have developed and used the property that
	\begin{equation}\label{eq:multradicals}
		\sqrt{ab} = \sqrt{a}\sqrt{b}
	\end{equation}
	where $a, b\geq 0$. In fact, this can be applied to $n^{th}$ roots as well.
\end{myDefinition} 

\begin{myexample}\label{ex:multradicalsdrillskill}
Simplify the following:
\drillandskill
\end{myexample}
\begin{multicols}{4}
	\begin{enumerate}
		\item $\sqrt{75}$ \solution{$=5\sqrt{3}$}
		\item $\sqrt{32}$ \solution{$=4\sqrt{2}$}
		\item $\sqrt{44}$ \solution{$=2\sqrt{11}$}
		\item $\sqrt{50}$ \solution{$=5\sqrt{2}$}
		\item $\sqrt{60}$ \solution{$=2\sqrt{3}\sqrt{5}$}
		\item $\sqrt{63}$ \solution{$=3\sqrt{7}$}
		\item $\sqrt{48}$ \solution{$=2\sqrt{3}$}
		\item $\sqrt{24}$ \solution{$=2\sqrt{2}\sqrt{3}$}
	\end{enumerate}
\end{multicols}

\subsection{Dividing radical expressions}
We know that
\[
	5^2 = 25, \quad {\mathrm{and}} \quad \sqrt{25}=5
\]
and that
\[
	6^2 = 36, \quad {\mathrm{and}} \quad \sqrt{36}=6
\]
Also, we know that when multiplying fractions
\[
	\frac{5}{6}\cdot \frac{5}{6} = \frac{25}{36}
\]
We can use this to help us simplify radical expressions involving division. For example, consider
simplifying the following
\begin{align*}
	\sqrt{\frac{25}{36}} & = \frac{\sqrt{25}}{\sqrt{36}} \\
	                     & = \frac{5}{6}                 
\end{align*} 
\begin{myDefinition}
	The equivalent result to that shown in \cref{eq:multradicals} is
	\[
		\sqrt{\frac{a}{b}} = \frac{\sqrt{a}}{\sqrt{b}}
	\]
	We put the restriction that $b\ne 0$. We now demonstrate this with examples.
\end{myDefinition} 

\begin{myexample}
Simplify the following
\begin{multicols}{2}
	\begin{enumerate}
		\item $\dd\sqrt{\frac{100}{49}}$
		\item $\dd\sqrt{\frac{30}{49}}$
	\end{enumerate} 
\end{multicols}
\end{myexample}
\begin{myProof}
	\begin{enumerate}
		\item 
		$\begin{aligned}[t]
			\sqrt{\frac{100}{49}} & =  \frac{\sqrt{100}}{\sqrt{49}} \\
			                      & = \frac{10}{7}                  
		\end{aligned}$
		\item 
		$\begin{aligned}[t]
			\sqrt{\frac{30}{49}} & =  \frac{\sqrt{30}}{\sqrt{49}} \\
			                     & =  \frac{\sqrt{30}}{7}         
		\end{aligned}$
	\end{enumerate} 
			
	Note that we can not simplify $\sqrt{30}$ any further, so we leave it as is.
				
\end{myProof} 

\begin{myexample}
Simplify the following
\begin{multicols}{2}
	\begin{enumerate}
		\item $\dd\sqrt[3]{\frac{8}{27}}$
		\item $\dd\sqrt[4]{\frac{64}{81}}$
	\end{enumerate}
\end{multicols}
{}
\end{myexample}
\begin{myProof}
	\begin{enumerate}
		\item 
		$\begin{aligned}[t]
			\sqrt[3]{\frac{8}{27}} & =  \frac{\sqrt[3]{8}}{\sqrt[3]{27}} \\
			                       & =  \frac{2}{3}                      
		\end{aligned}$
		\item 
		$\begin{aligned}[t]
			\sqrt[4]{\frac{64}{81}} & =  \frac{\sqrt[4]{64}}{\sqrt[4]{81}} \\
			                        & =  \frac{4}{3}                       
		\end{aligned}$
	\end{enumerate} 
\end{myProof}

\begin{myexample}
Simplify the following
\drillandskill
\end{myexample}
\begin{multicols}{2}
	\begin{enumerate}
		\item $\dd\sqrt{\frac{25}{7}}$ \solution{$\dd=\frac{5}{\sqrt{7}}$}
		\item $\dd\frac{\sqrt{25}}{\sqrt{7}}$ \solution{$\dd=\frac{5}{\sqrt{7}}$}
		\item $\dd\frac{\sqrt{21}}{\sqrt{3}}$ \solution{$\dd=\sqrt{7}$}
		\item $\dd\frac{\sqrt{27}}{\sqrt{3}}$ \solution{$\dd=3$}
		\item $\dd\frac{\sqrt{35}}{\sqrt{5}}$ \solution{$\dd=\sqrt{7}$}
		\item $\dd\frac{\sqrt{45}}{\sqrt{3}}$ \solution{$\dd=\frac{3\sqrt{5}}{\sqrt{3}}$}
		\item $\dd\frac{\sqrt{15}}{\sqrt{5}}$ \solution{$\dd=\sqrt{3}$}
		\item $\dd\frac{\sqrt{55}}{\sqrt{3}}$ \solution{$\dd=\frac{\sqrt{5}\sqrt{11}}{\sqrt{3}}$}
	\end{enumerate}
\end{multicols}

\subsection{Simplifying radical expressions by combining like terms}
Sometimes radical expressions may be simplified by adding or subtracting. We can combine like
terms just as we would do with $x$ or $x^2$ terms for example. It is very important to note
that we can only add and subtract like radicals- for example the \gls{expression}
\[
	\sqrt{x}+ \sqrt[3]{x}
\]
can not be simplified any further.

\begin{myexample}
Combine
\[
	\sqrt{3}+5\sqrt{3} -2\sqrt{3}
\]
\end{myexample}
\begin{myProof}
	Are they all like radicals? Yes, so we can combine the coefficients. Remember that the \gls{coefficient}
	of the first term is 1 even though it is not seen there.
	\begin{align*}
		\sqrt{3}+5\sqrt{3}-2\sqrt{3} & =  (1+5-2)\sqrt{3} \\
		                             & =  4\sqrt{3}       
	\end{align*} 
	Even though we do not usually write the parenthesis as we have done in the above, this helps to 
	visualize the thought process that goes into calculating the answer. 
\end{myProof} 


\begin{myexample}
Simplify
\[
	2\sqrt{5}+3\sqrt{2}-4\sqrt{5}+6\sqrt{2}
\]
\end{myexample}
\begin{myProof}
	Remember to combine the only like radicals. The way we do this relies on the commutative
	property of addition- in other words we can rearrange the terms into any order
	\[
		2\sqrt{5} - 4\sqrt{5}+3\sqrt{2}+6\sqrt{2} = -2\sqrt{5}+9\sqrt{2}
	\]
	{}
\end{myProof} 
In the next examples we will combine the techniques that we have discussed so far.

\begin{myexample}
Simplify 
\[
	2\sqrt{18}+ 4\sqrt{50}
\]
\end{myexample}
\begin{myProof}
	Notice first of all that the radicands (the numbers inside the radical symbol) are not the same. This
	does not necessarily mean that the expression can not be simplified. We factor each radicand into
	the product of perfect squares, and simplify as follows
	\begin{align*}
		2\sqrt{18}+ 4\sqrt{50} & =  2\sqrt{9\cdot 2}+ 4 \sqrt{25\cdot 2}             \\
		                       & =  2\sqrt{9}\sqrt{2}+4\sqrt{25}\sqrt{2}             \\
		                       & =  2\cdot 3 \cdot \sqrt{2} + 4\cdot 5\cdot \sqrt{2} \\
		                       & =  6\sqrt{2} + 20 \sqrt{2}                          \\
		                       & =  26\sqrt{2}                                       
	\end{align*} 
	A safe technique is to work with each term individually, and then to combine like terms at the end.
\end{myProof} 
We have so far concentrated on factoring numbers under the radical symbol by factoring into perfect
squares or perfect cubes, etc. Another technique is to factor the radicand into the product of primes. 
Either way is perfectly acceptable, and you should pick whichever suits you best. The next example demonstrates
factoring into the product of primes. 

\begin{myexample}
Combine
\[
	5\sqrt{75}-4\sqrt{12}+3\sqrt{8}
\]
\end{myexample}
\begin{myProof}
	\begin{align*}
		5\sqrt{75}-4\sqrt{12}+3\sqrt{8} & =  5\sqrt{5\cdot 5\cdot 3 }	-4\sqrt{2\cdot 2 \cdot 3  }+3\sqrt{2\cdot 2 \cdot 2} \\  
		                                & =  5\cdot 5\sqrt{3}-4\cdot 2\sqrt{3} + 3\cdot 2\sqrt{2}                          \\
		                                & =  25\sqrt{3} - 8 \sqrt{3} + 6\sqrt{2}                                           \\	
		                                & =  17\sqrt{3}+6\sqrt{2}                                                          
	\end{align*} 
	Notice that the last line in the above does not contain like radicals, and can not be simplified any further.
\end{myProof} 

\begin{myexample}
Simplify
\[
	4\sqrt[3]{54} - 2\sqrt[3]{64}
\]
\end{myexample}
\begin{myProof}
	We begin by noting that $54=27\cdot 2$, and that $\sqrt[3]{64}=4$, which helps
	us as follows
	\begin{align*}
		4\sqrt[3]{54} - 2\sqrt[3]{64} & =  4\sqrt[3]{27 \cdot 2} - 2(4)       \\
		                              & =  4\sqrt[3]{27}\cdot \sqrt[3]{2} - 8 \\
		                              & =  12 \sqrt[3]{2}-8                   
	\end{align*}
	We can not simplify this expression any further.
\end{myProof}

\begin{myexample}
Find the following square roots- if the answer is not a real number, then say so
\drillandskill
\end{myexample}
\begin{multicols}{2}
	\begin{enumerate}
		\item $\sqrt{\frac{1}{4}}$ \solution{$=\frac{1}{2}$}
		\item $\sqrt{\frac{1}{9}}$ \solution{$=\frac{1}{3}$}
		\item $\sqrt{\frac{25}{36}}$ \solution{$=\frac{5}{6}$}
		\item $\sqrt{-\frac{1}{4}}$ \solution{not a real number}
		\item $\sqrt{9}+\sqrt{16}$ \solution{$=7$}
		\item $\sqrt{9+16}$ \solution{$=5$}
		\item $\sqrt{-\frac{1}{2}}$ \solution{not a real number}
		\item $\sqrt{3-17}$ \solution{not a real number}
	\end{enumerate}
\end{multicols}

\section{More on multiplying radicals}
\textref{9.3}{567}%
We saw in \vref{eq:multradicals} that
\[
	\sqrt{a}\sqrt{b} = \sqrt{ab}
\]
where $a, b\geq 0$. We have used this property to help us simplify radical 
expressions by factoring numbers into the product of perfect squares (see 
\cref{ex:multradicals,ex:multradicalsdrillskill}).

We can deduce that
\begin{myDefinition}
	For any $a\geq 0$
	\[
		\sqrt{a}\sqrt{a} = a
	\]
\end{myDefinition} 

\begin{myexample}
\drillandskill
Simplify the following:
\end{myexample}
\begin{multicols}{2}
	\begin{enumerate}
		\item $\sqrt{5}\sqrt{5} = 5$
		\item $\sqrt{7}\sqrt{7} = 7$
		\item $\sqrt{13}\sqrt{13}=13$
		\item $\sqrt{1045}\sqrt{1045}=1045$
	\end{enumerate} 
\end{multicols}

\subsection{FOILing with radicals}
We can operate with radicals using the same algebraic techniques we have 
developed so far.

\begin{myexample}
Perform the following multiplication
\begin{multicols}{2}
	\begin{enumerate}
		\item $\sqrt{3}(5+\sqrt{7})$
		\item $(\sqrt{2}+3)(\sqrt{5}+7)$
	\end{enumerate} 
\end{multicols}
\end{myexample}
\begin{myProof}
	\begin{enumerate}
		\item $ \sqrt{3}(5+\sqrt{7})  =  5\sqrt{3}+\sqrt{3}\sqrt{7} $
		\item We use the \gls{FOIL} method
		\[ 
			(\sqrt{2}+3)(\sqrt{5}+7)  =  \sqrt{2}\sqrt{5}+7\sqrt{2}+3\sqrt{5}+21
		\]
	\end{enumerate} 
\end{myProof} 

\begin{myexample}\label{ex:conjugatemult}
Perform the following multiplication:
\begin{multicols}{3}
	\begin{enumerate}
		\item $(2+\sqrt{3})(2-\sqrt{3})$
		\item $(5-\sqrt{7})(5+\sqrt{7})$
		\item $(\sqrt{11}+\sqrt{5})(\sqrt{11}-\sqrt{5})$
	\end{enumerate} 
\end{multicols}
\end{myexample}
\begin{myProof}
	We will use the FOIL method.
	\begin{enumerate}
		\item $
		\begin{aligned}[t]
			(2+\sqrt{3})(2-\sqrt{3}) & =  4 - 2\sqrt{3} + 2\sqrt{3} -3 \\
			                         & =  4-3                          \\
			                         & =  1                            
		\end{aligned}
		$
		\item $
		\begin{aligned}[t]
			(5-\sqrt{7})(5+\sqrt{7}) & =  25 + 5\sqrt{7} - 5\sqrt{7} -7 \\
			                         & =  25 -7                         \\
			                         & =  18                            
		\end{aligned}
		$
		\item $
		\begin{aligned}[t]
			(\sqrt{11}+\sqrt{5})(\sqrt{11}-\sqrt{5}) & =  11 -\sqrt{5}\sqrt{11} +\sqrt{5}\sqrt{11} - 5 \\
			                                         & =  11-5                                         \\
			                                         & =  6                                            
		\end{aligned}
		$
	\end{enumerate} 
\end{myProof} 

What do you notice about all of the solutions in \cref{ex:conjugatemult}? We note that each \gls{solution} does 
not contain a radical. Notice also that each problem involved the multiplication of the sum and difference of 
two terms. This leads us to the definition of the {\em conjugate}.

\begin{myDefinition}\label{def:conjugate}
	The conjugate of
	\[
		\sqrt{a}-\sqrt{b}
	\]
	is 
	\[
		\sqrt{a}+\sqrt{b}
	\]
	The conjugate is useful because
	\[
		(\sqrt{a}+\sqrt{b})(\sqrt{a}-\sqrt{b}) = a-b
	\]
\end{myDefinition} 

\begin{myexample}
Find the conjugate of each of the following:
\begin{multicols}{4}
	\begin{enumerate}
		\item $2+\sqrt{5}$
		\item $3-\sqrt{2}$
		\item $9+\sqrt{17}$
		\item $\sqrt{19}+10$
	\end{enumerate} 
\end{multicols}
\end{myexample}
\begin{myProof} 
	This is a simple procedure: change the sign between the terms.
	\begin{multicols}{4}
		\begin{enumerate}
			\item $2-\sqrt{5}$
			\item $3+\sqrt{2}$
			\item $9-\sqrt{17}$
			\item $\sqrt{19}-10$
		\end{enumerate} 
	\end{multicols}
\end{myProof} 

\begin{myexample}
\drillandskill
Perform the following multiplication:
\begin{multicols}{2}
	\begin{enumerate}
		\item $\sqrt{2}(11+\sqrt{3})$\solution{$=11\sqrt{2}+\sqrt{6}$}	
		\item $\sqrt{5}(11-\sqrt{3})$\solution{$=11\sqrt{5}-\sqrt{3}\sqrt{5}$}	
		\item $-\sqrt{3}(4+\sqrt{7})$\solution{$=-4\sqrt{3}-\sqrt{3}\sqrt{7}$}	
		\item $\sqrt{7}(\sqrt{19}+\sqrt{7})$\solution{$=\sqrt{7}\sqrt{19}+7$}	
		\item $(\sqrt{5}+2)(\sqrt{7}-1)$\solution{$=\sqrt{35}-\sqrt{5}+2\sqrt{7}-2$}
		\item $(\sqrt{3}+\sqrt{2})(\sqrt{3}+\sqrt{2})$\solution{$=5+2\sqrt{6}$}
		\item $(\sqrt{3}-\sqrt{2})(\sqrt{3}-\sqrt{2})$\solution{$=5-2\sqrt{6}$}
		\item $(\sqrt{3}+\sqrt{2})(\sqrt{3}-\sqrt{2})$\solution{$=1$}
	\end{enumerate}
\end{multicols}
\end{myexample}
