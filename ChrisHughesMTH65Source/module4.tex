%+*** mainfile.tex
% !arara: pdflatex: { files: [ mainfile.tex ] }
% arara: makechapters: { files:[mainfile], items: [module4], makeChapGlossaries: no}
% !arara: indent: { overwrite: on, trace: yes, localSettings: on}

\chapter{Quotient rules}
\minitoc
We have previously considered \gls{polynomial} multiplication, and evaluation. In particular,
we have studied the \gls{FOIL} method for multiplying two binomials, and we used it when 
dealing with polynomials in one and more than \gls{variable}. In this module we will study
how to divide polynomials.


\section{The quotient rule for exponents}
\textref{6.5}{376}%
Consider the quotient of two exponential expressions, such as 
\begin{align*}
	\frac{2^7}{2^3} & =	 \frac{2\cdot 2\cdot 2\cdot 2\cdot 2\cdot 2\cdot 2 }{2\cdot 2\cdot  2} \\
	                & =	 2 \cdot 2\cdot 2 \cdot 2                                              \\
	                & =	 2^4                                                                   
\end{align*} 
We have therefore shown that
\[
	\frac{2^7}{2^3} = 2^4
\]
We can find the exponent, $4$, on the quotient by {\em subtracting} the original exponents. Remember
that when multiplying two exponential expressions with the same base, we added the exponents.

\begin{myDefinition}
	Given any real number $b\ne 0$, and any integers $m$, $n$, we can state the quotient rule as
	\[
		\frac{b^m}{b^n} = b^{m-n}
	\]
	When dividing exponential expressions with the same nonzero base, subtract the
	exponent in the denominator from the exponent in the numerator. Use this
	difference as the exponent of the common base.
\end{myDefinition}

\begin{myexample}
Divide each \gls{expression} using the quotient rule, expressing any numerical answers in exponential form.
\begin{multicols}{3}
	\begin{enumerate}
		\item $\dd\frac{5^{20}}{5^{15}}$
		\item $\dd\frac{y^{15}}{y^3}$
		\item $\dd\frac{x^{200}y^{40}}{x^{25}y^{10}}$
	\end{enumerate} 
\end{multicols}
\end{myexample}
\begin{myProof}
	\begin{enumerate}
		\item 
		$\begin{aligned}[t]
			\frac{5^{20}}{5^{15}} & =		5^{20-15} \\
			                      & =		5^5       
		\end{aligned} $
		\item 
		$\begin{aligned}[t]
			\frac{y^{15}}{y^3} & =		y^{15-3} \\
			                   & =		y^{12}   
		\end{aligned}$ 
		\item Remember to analyze which terms can be simplified with one another
		\begin{align*}
			\frac{x^{200}y^{40}}{x^{25}y^{10}} & =		x^{200-25}y^{40-10} \\
			                                   & =		x^{175}y^{30}       
		\end{align*} 
	\end{enumerate} 
\end{myProof} 

\begin{myexample}
\Gls{simplify} the following.
\drillandskill
\end{myexample}
\begin{multicols}{4}
	\begin{enumerate}
		\item $\dd\frac{2^8}{2^3}$\solution{$=2^5$}
		\item $\dd\frac{x^4}{x^2}$\solution{$=x^2$}
		\item $\dd\frac{x^9}{x^3}$\solution{$=x^6$}
		\item $\dd\frac{y^{10}}{y^{2}}$\solution{$=y^8$}
		\item $\dd \frac{4x^5}{2x^3}$\solution{$=2x^2$}
		\item $\dd \frac{10y^7}{2y^2}$\solution{$=5y^5$}
		\item $\dd -\frac{40x^9}{10x^2}$\solution{$=-4x^7$}
		\item $\dd \frac{50x^2}{5x}$\solution{$=10x$}
	\end{enumerate}
\end{multicols}

Note that we can only cancel {\em factors} of the numerator and denominator. Consider the following example
\begin{align*}
	\frac{2^3+2^4}{2^3} & =	 \frac{2\cdot 2\cdot 2 + 2\cdot2\cdot2\cdot2}{2\cdot 2\cdot 2} \\
	                    & \ne	 2\cdot 2\cdot 2\cdot 2                                      
\end{align*} 
We will discuss later how to divide expressions that involve more than one term in either the numerator
or denominator. For the moment, our expressions will involve only factors.

\section{The zero exponent}
What is the meaning of zero as an exponent? We certainly know that the following is true
\[
	\frac{b^4}{b^4} = 1
\]
We also know that from the quotient rule
\begin{align*}
	\frac{b^4}{b^4} & =		b^{4-4} \\
	                & =		b^0     
\end{align*} 
These two expressions {\em must} be equivalent, and we therefore conclude
that
\[\
	b^0=1
\]
for any real number $b$.
\begin{myexample}
Simplify the following
\drillandskill
\end{myexample}
\begin{multicols}{4}
	\begin{enumerate}
		\item $x^0$\solution{$=1$}
		\item $73^0$\solution{$=1$}
		\item $-(2^2)^0$\solution{$=-1$}
		\item $(-2)^0$\solution{$=1$}
		\item $-x^0$\solution{$=-1$}
		\item $-3(2^0)$\solution{$=-3$}
		\item $-4(-2^0)$\solution{$=4$}
		\item $-5(-2^2)^0$\solution{$=-5$}
	\end{enumerate}
\end{multicols}

\section{The Quotients to powers rule}
Recall that when a (grouped) product is raised to a power, we raise every \gls{factor} in the product to the power
\[
	(ab)^n = a^n b^n
\]
Note that without the parenthesis the product $ab^n$  means something different, as only $b$ is raised to the
$n$th power.

We can apply this principle to quotients raised to a power. If $a$ and $b$ are real numbers
and $b$ is non zero, then
\[
	\left( \frac{a}{b}\right)^n = \frac{a^n}{b^n}
\]
When a (grouped) quotient is raised to a power, we raise the numerator to the power and 
divide by the denominator raised to the same power.

\begin{myexample}
Simplify the following
\begin{multicols}{2}
	\begin{enumerate}
		\item $\dd\left(\frac{1}{2}\right)^3$
		\item $\dd\left(\frac{x}{y}\right)^5$
	\end{enumerate} 
\end{multicols}
\end{myexample}
\begin{myProof}
	\begin{enumerate}
		\item 
		$\begin{aligned}[t]
			\left(\frac{1}{2}\right)^3 & =		\frac{1^3}{2^3} \\
			                           & =		\frac{1}{2^3}   \\
			                           & =		\frac{1}{8}     
		\end{aligned}	$
		\item 
		$\begin{aligned}[t]
			\left(\frac{x}{y}\right)^5 & =		\frac{x^5}{y^5} \\
		\end{aligned}$
	\end{enumerate} 
\end{myProof} 
Note that without the parenthesis, the quotient $\dd\frac{a^n}{b}$ means something different to $\dd\left(\frac{a}{b}\right)^n$. In
$\dd\frac{a^n}{b}$, it is only $a$ that is raised to the $n$th power.

\begin{myexample}
Simplify the following
\drillandskill
\end{myexample}
\begin{multicols}{2}
	\begin{enumerate}
		\item $\dd\left(\frac{x}{2}\right)^3$\solution{$\dd=\frac{x^3}{8}$}
		\item $\dd\left(\frac{x^3}{2}\right)^3$\solution{$\dd=\frac{x^9}{8}$}
		\item $\dd\left(-\frac{x}{5}\right)^2$\solution{$\dd=\frac{x^2}{25}$}
		\item $\dd\left(-\frac{2x}{3}\right)^2$\solution{$\dd=\frac{4x^2}{9}$}
		\item $\dd\left(-\frac{2x}{3}\right)^3$\solution{$\dd=-\frac{8x^3}{27}$}
	\end{enumerate}
\end{multicols}

\section{Dividing by monomials}
Understanding and remembering how to divide polynomials heavily depends on your ability to understand
the difference between adding and multiplying fractions.

For example, consider the following; the first is a fraction multiplication, 
and the second is a fraction addition
\begin{multicols}{2}
	\begin{itemize}
		\item $\dd\frac{2}{5}\cdot \frac{7}{5}= \frac{2\cdot 7}{5\cdot 5}$
		\item $\dd\frac{2}{5}+\frac{7}{5} = \frac{2+7}{5}$
	\end{itemize} 
\end{multicols}
Comparing these with the following examples, we see why we must distribute $4x^2$ in the 2nd example, but
not the 1st:
\begin{itemize}
	\item $\dd\frac{-12x^8}{4x^2}$ is really $\dd\frac{-12}{4}\cdot \frac{x^8}{x^2}$
	\item $\dd\frac{-12+x^8}{4x^2}$ is really $\dd\frac{-12}{4x^2}+\frac{x^8}{4x^2}$
\end{itemize} 

\begin{myexample}
Simplify the following
\begin{multicols}{2}
	\begin{enumerate}
		\item $\dd\frac{-12x^8}{4x^2}$
		\item $\dd\frac{-12+x^8}{4x^2}$
	\end{enumerate} 
\end{multicols}
\end{myexample}
\begin{myProof}
	\begin{enumerate}
		\item  
		$\begin{aligned}[t]
			\frac{-12x^8}{4x^2} & =		\frac{-12}{4}\cdot\frac{x^8}{x^2} \\
			                    & =		-3x^{8-2}                         \\
			                    & =		-3x^6                             
		\end{aligned}$
		\item 
		$\begin{aligned}[t]
			\frac{-12+x^8}{4x^2} & =		\frac{-12}{4x^2}+\frac{x^8}{4x^2}                               \\
			                     & =		\frac{-12}{4}\cdot\frac{1}{x^2}+\frac{1}{4}\cdot\frac{x^8}{x^2} \\
			                     & =		-3 \frac{1}{x^2} + \frac{1}{4}x^{8-2}                           \\
			                     & =		\frac{-3}{1}\cdot\frac{1}{x^2}+\frac{1}{4}\cdot\frac{x^6}{1}    \\
			                     & =		-\frac{3}{x^2}+\frac{x^6}{4}                                    
		\end{aligned}$
	\end{enumerate} 
\end{myProof} 

To summarize: 
\begin{itemize}
	\item To divide monomials, divide the coefficients and then divide the variables. Use the quotient
	rule for exponents to divide the variables and subtract the exponents
	\item To divide a polynomial by a \gls{monomial}, divide each term of the polynomial by the monomial
\end{itemize} 

\begin{myexample}
Simplify the following
\drillandskill
\end{myexample}

\begin{multicols}{2}
	\begin{enumerate}
		\item $\dd \frac{x^7+x^2}{x}$\solution{$=x^6+x$}
		\item $\dd \frac{3x^8+x^{10}}{x^4}$\solution{$=3x^4+x^6$}
		\item $\dd \frac{-3x^8+x^6+x^2}{x^2}$\solution{$=-3x^6+x^4+1$}
		\item $\dd \frac{x^9+x^2+3x}{x}$\solution{$=x^8+x+3$}
	\end{enumerate}
\end{multicols}

\section{Negative exponents and scientific notation}
\textref{6.7}{393}%
In this section we will look
at {\em negative exponents} and {\em scientific notation}.

\subsection{Negative exponents}
We define negative exponents as
\[
	a^{-n} = \frac{1}{a^n}
\]	
where (for this class) $n$ is an integer.

\begin{myexample}\label{ex:negexponnumeric}
Write the following without any negative exponents
\begin{multicols}{5}
	\begin{enumerate}
		\item $2^{-1}$
		\item $2^{-2}$
		\item $3^{-3}$
		\item $3^{-4}$
		\item $\left(\frac{1}{3}\right)^{-2}$
	\end{enumerate} 
\end{multicols}
\end{myexample}
\begin{myProof}
	We use the above definition to re-write these expressions
	\begin{enumerate}
		\item 
		$\begin{aligned}[t]
			2^{-1} & =  \frac{1}{2^1} \\
			       & =  \frac{1}{2}   
		\end{aligned}$
		\item 
		$\begin{aligned}[t]
			2^{-2} & =  \frac{1}{2^2} \\
			       & =  \frac{1}{4}   
		\end{aligned}$
		\item 
		$\begin{aligned}[t]
			3^{-3} & =  \frac{1}{3^3} \\
			       & =  \frac{1}{27}  
		\end{aligned}$
		\item 
		$\begin{aligned}[t]
			3^{-4} & =  \frac{1}{3^4} \\
			       & =  \frac{1}{81}  
		\end{aligned}$ 
		\item 
		$\begin{aligned}[t]
			\left(\frac{1}{3}\right)^{-2} & =  \frac{1}{\left(\frac{1}{3}\right)^2} \\
			                              & =  \frac{1}{\frac{1}{9}}                \\
			                              & =  9                                    \\
			                              & =  \left(\frac{3}{1}\right)^2           
		\end{aligned}$ 
	\end{enumerate} 
	Notice in the last part that we {\em flip the fraction} when raising a fraction to a negative exponent.
\end{myProof}

\begin{myexample}\label{ex:negexponnumericds}
Write each of the following using only positive exponents.
\drillandskill
		
\end{myexample}

\begin{multicols}{2}
	\begin{enumerate}
		\item $4^{-1}$ \solution{$\dd=\frac{1}{4}$}
		\item $6^{-2}$\solution{$\dd=\frac{1}{36}$}
		\item $7^{-4}$\solution{$\dd=\frac{1}{7^4}$}
		\item $9^{-15}$\solution{$\dd=\frac{1}{9^{15}}$}\\
		\item $\dd\left(\frac{1}{4}\right)^{-2}$\solution{$\dd=16$}
		\item $\dd\left(\frac{2}{3}\right)^{-2}$\solution{$\dd=\frac{9}{4}$}
		\item $\dd\left(\frac{2}{3}\right)^{-3}$\solution{$\dd=\frac{27}{8}$}
		\item $\dd\left(\frac{4}{5}\right)^{-2}$\solution{$\dd=\frac{25}{16}$}
	\end{enumerate}
\end{multicols}

{\em More advanced}
\begin{multicols}{2}
	\begin{enumerate}
		\item $\left(-\frac{1}{4}\right)^{-2}$\solution{$\dd=16$}
		\item $\left(-\frac{2}{3}\right)^{-2}$\solution{$\dd=\frac{9}{4}$}
		\item $\left(-\frac{2}{3}\right)^{-3}$\solution{$\dd=-\frac{27}{8}$}
		\item $\left(-\frac{4}{5}\right)^{-2}$\solution{$\dd=\frac{25}{16}$}
		\item $(x^3)^{-5}$\solution{$\dd=\frac{1}{x^{15}}$}
		\item $(x^{-5})^3$\solution{$\dd=\frac{1}{x^{15}}$}
		\item $(-x^2)^{-4}$\solution{$\dd=\frac{1}{x^8}$}
		\item $(-x^2)^{-3}$\solution{$\dd=-\frac{1}{x^6}$}
	\end{enumerate}
\end{multicols}

\begin{myexample}
Use the techniques from \cref{ex:negexponnumeric,ex:negexponnumericds} to simplify the following expressions.
\begin{multicols}{2}
\begin{enumerate}
	\item $\left(\dfrac{2}{y^3}\right)^{-4}$
    \item $\left( \dfrac{s^3}{t^5} \right)^{-6}$
\end{enumerate}
\end{multicols}
\end{myexample}
\begin{myProof}
  \begin{enumerate}
    \item 
      $\begin{aligned}[t]
	\left(\dfrac{2}{y^3}\right)^{-4} & = \left(\dfrac{y^3}{2}\right)^{4} \\
    & = \dfrac{y^{3\cdot 4}}{2^4}\\
    &= \dfrac{y^{12}}{16}
      \end{aligned}$

      Alternatively, you might have preferred to write
      \begin{align*}
        \left(\frac{2}{y^3}\right)^{-4} & = \left(\frac{2^{-4}}{y^{3\cdot -4}}\right) \\
        & = \frac{\frac{1}{16}}{y^{-12}}\\
    &= \frac{y^{12}}{16}
      \end{align*}
    \item
      $
      \begin{aligned}[t]
    \left( \dfrac{s^3}{t^5} \right)^{-6} & =  \left( \dfrac{t^5}{s^3} \right)^{6}\\
    & = \dfrac{t^{5\cdot 6}}{s^{3\cdot 6}} \\
    & = \dfrac{t^{30}}{s^{18}}
      \end{aligned}
      $
  \end{enumerate}
\end{myProof}

\begin{myexample}
  Simplify each of the given expressions as much as possible.
\drillandskill
\begin{multicols}{2}
	\begin{enumerate}
		\item $\dd\left( \frac{x^2}{y^3}\right)^{-2}$\solution{$\dd=\frac{y^6}{x^4}$}
		\item $\dd\left( \frac{y^2}{y^3}\right)^{-2}$\solution{$\dd=y^2$}
		\item $\dd\left( \frac{x^2z^7}{y^3}\right)^{-2}$\solution{$\dd=\frac{y^6}{x^4z^{14}}$}
		\item $\dd\left( \frac{-4x^3}{y^3}\right)^{-2}$\solution{$\dd=\frac{y^6}{16x^6}$}
	\end{enumerate}
\end{multicols}
\end{myexample}

\subsection{Scientific notation for numbers with absolute value $>10$}
We will now use our knowledge of exponents, both positive and negative, to write numbers that are very small or
very large in a neat and useful way. 

Consider, for example, the distance between the Sun and Pluto, which is
5,906,000,000km. This is a huge distance \footnote{{Of course, astronomically speaking the Sun is considered quite close to us.}}, and if 
we wanted to reference it regularly then it would be very cumbersome to write 5,906,000,000km every single time. Scientific notation is a short
hand way of writing powers of 10, and we will use it to simplify large numbers such as the one just considered.

We begin with the following 
\begin{align*}
	10    & =  1 \times 10   \\
	100   & =  1 \times 10^2 \\
	1000  & =  1 \times 10^3 \\
	10000 & =  1 \times 10^4 \\
\end{align*}
Notice here that the number of 0s {\em after} the number 10 on the left hand side is equal to the power of 10 on 
the right hand side.

The following guidelines may help you in converting numbers into scientific notation for numbers whose absolute
values is {\em greater than or equal to 10}.
\begin{itemize}
	\item To write a number whose absolute value is 10 or greater in scientific notation, write a product of
	a numerical factor and an exponential expression.
	\item The numerical factor is determined by moving the decimal \gls{point} to the left so that the resulting
	number is between 1 and 10, including 1.
	\item The exponential expression consists of the base 10 and a positive exponent that is the number of places the
	decimal was moved.
\end{itemize} 

\begin{myexample}
Write the following numbers in scientific notation
\begin{multicols}{2}
	\begin{enumerate}
		\item 5,906,000,000
		\item 32,100,000
	\end{enumerate} 
\end{multicols}
\end{myexample}
\begin{myProof}
	\begin{enumerate}
		\item The absolute value of this number is clearly greater than 10. We write this in scientific notation by
		moving the decimal point from the right until we have a number between 1 and 10 (in this case 5). We 
		then count the number of places that we had to move the decimal, and this will be the power of 10. So, 
		\[
			5,906,000,000 = 5.906 \times 10^9
		\]
		When you try this type of calculation in your calculator, you may find that it displays the result as
		\[
			5.906 E 9
		\]	
		You can read the `E' as `times 10 to the power of'. 
		\item Proceeding as in part a), 
		\begin{align*}
			32,100,000 & =  3.21\times 10^7 \\
			           & =  3.21 E7         
		\end{align*} 
	\end{enumerate} 
	{}
\end{myProof}

\subsection{Scientific notation for numbers with absolute value $<10$}
In a similar way, we have
\begin{align*}
	0.1    & =  1 \times 10^{-1} \\
	0.01   & =  1 \times 10^{-2} \\
	0.001  & =  1 \times 10^{-3} \\
	0.0001 & =  1 \times 10^{-4} \\
\end{align*} 
Notice here that the power of 10 decreases as the number decreases in magnitude. 

\begin{itemize}
	\item To write a number whose absolute value is less than 1 in scientific notation, write a product of 
	a numerical factor and an exponential expression.
	\item The numerical factor is determined by moving the decimal point to the right so that the resulting
	number is between 1 and 10, including 1.
	\item The exponential expression consists of the base 10 and a negative exponent that has an absolute
	value of the number of places the decimal was moved.
\end{itemize} 

\begin{myexample}
Write the following numbers in scientific notation
\begin{multicols}{2}
	\begin{enumerate}
		\item 0.00012
		\item -0.005
	\end{enumerate} 
\end{multicols}
\end{myexample}
\begin{myProof}
	\begin{enumerate}
		\item
          $\begin{aligned}[t]
			0.00012 & =  1.2\times 10^{-4} \\
			        & =  1.2 E -4          
                  \end{aligned}$
		\item
          $\begin{aligned}[t]
			-0.005 & =  -5 \times 10^{-3} \\
			       & =  -5E-3             
                 \end{aligned}$
	\end{enumerate} 
	{}
\end{myProof} 

\begin{myexample}
Simplify the following
\drillandskill
\end{myexample}

\begin{multicols}{2}
	\begin{enumerate}
		\item $32,400$\solution{$=3.24 \times 10^{4}$}
		\item $713$\solution{$=7.13\times 10^2$}
		\item $9876$\solution{$=9.876\times 10^3$}
		\item $12000$\solution{$=1.2\times 10^4$}
		\item $0.001$\solution{$=1\times 10^{-3}$}
		\item $0.0005$\solution{$=5\times 10^{-4}$}
		\item $-0.09$\solution{$=-9\times 10^{-2}$}
		\item $0.000012$\solution{$=1.2\times 10^{-5}$}
	\end{enumerate}
\end{multicols}

\subsection{Converting from scientific notation to standard notation}
Given a number in scientific notation, we first need to determine if the
exponent is negative or positive:
\begin{itemize}
	\item if it is positive then move the decimal point in the numerical factor to the right by the number of 
	places denoted by the exponent. For example
	\[
		3.00\times 10^8 = 300,000,000
	\]
	\item if is negative then move the decimal point in the numerical factor to the 
	left the number of places denoted by the exponent. For example
	\[
		7.4\times 10^{-5} = 0.000074
	\]
\end{itemize} 

