% Mainfile:
%%%***++ mainfile.tex
% arara: makechapters: { files: [ mainfile], items: [module5] }
% !arara: indent: { overwrite: on, trace: yes}
\chapter{Scientific notation and factoring}
\minitoc

\section{Introduction to factoring}
\textref{7.1}{414}%
So far in this class we have studied polynomials; in particular we have looked at how to multiply
polynomials. Our goal for the remainder of this module will be to study how we can `go the other way'. The question we ask ourselves is: given a
\gls{polynomial}, can we find an equivalent \gls{expression} but in factored form? There are a few techniques that
will help us in this study, and we demonstrate by example.

\begin{myexample}\label{ex:motivationfacto}
Perform the following multiplication.
\begin{multicols}{2}
	\begin{enumerate}
		\item $3(x+1)$
		\item $x^2(x^3-4x+5)$
	\end{enumerate} 
\end{multicols}
\end{myexample}
\begin{myProof}
	This example is designed as a warm up- our goal in the content after 
	this example will be to `go the other way'.
	\begin{enumerate}
		\item $3(x+1) = 3x+3$
		\item $x^2(x^3-4x+5) = x^5-4x^3+5x^2$
	\end{enumerate} 
\end{myProof} 

\Cref{ex:motivationfacto} reminds us of the distributive property
of multiplication. Now consider the expression
\[
	4x+2
\]
The question we ask is: are there any common factors to both terms? The
answer is yes: $2$. This means that we can reverse the multiplication procedure
and write the expression as follows
\[
	4x+2 = 2(2x+1)
\]

The process that we have just demonstrated is a very simple example of factoring. The process
is further demonstrated in the next example.

\begin{myexample}
\Gls{factor} each of the given expressions.
\drillandskill
\end{myexample}

{\em Introductory}
\begin{multicols}{2}
	\begin{enumerate}
		\item $4x+16$\solution{$=4(x+4)$}
		\item $5x+30$\solution{$=5(x+6)$}
		\item $-2x+10$\solution{$=-2(x-5)$}
		\item $-2x-10$\solution{$=-2(x+5)$}
	\end{enumerate}
\end{multicols}
{\em More advanced}
\begin{multicols}{2}
	\begin{enumerate}
		\item $7x^2+49x$\solution{$=7x(x+7)$}
		\item $-7x^2+49x$\solution{$=-7x(x-7)$}
		\item $-7x^2-49x$\solution{$=-7x(x+7)$}
		\item $3x^5 + 15x^3$\solution{$=3x^3(x^2+5)$}
		\item $14w^8 - 7w^9$\solution{$=7w^8(2w-1)$}
		\item $-32y^8 - 64y^2$\solution{$=-32y^2(y^6+2)$}
	\end{enumerate}
\end{multicols}
{\em Further advanced}
\begin{multicols}{2}
	\begin{enumerate}
		\item $2x^2+4x+2$\solution{$=2(x^2+2x+1)$}
		\item $2x^2+4x-2$\solution{$=2(x^2+2x-1)$}
		\item $4x^3+4x+16$\solution{$=4(x^3+4x+4)$}
		\item $4x^3+4x^2+16x$\solution{$=4x(x^2+x+4)$}
		\item $5w^4 - 25w^2 - 5w$\solution{$=5w(w^3-5w-1)$}
		\item $5w^4 - 25w^2 + 50w$\solution{$=5w(w^3-5w+10)$}
		\item $6w^2 - 30w - 3$\solution{$=3(2w^2-10w-1)$}
		\item $-3x^2-9x+6$\solution{$=-3(x^2+3x-2)$}
	\end{enumerate}
\end{multicols}

{\em Examples with 2 variables}
\begin{multicols}{2}
	\begin{enumerate}
		\item $4xy-2x$\solution{$=2x(2y-1)$}
		\item $4xy-2y$\solution{$=2y(2x-1)$}
		\item $8x^2y-2x$\solution{$=2x(4xy-1)$}
		\item $8x^2y-2xy$\solution{$=2xy(4x-1)$}
		\item $16x^2y^2-2xy$\solution{$=2xy(8xy-1)$}
		\item $32x^2y^2-4x^2y$\solution{$=4x^2y(8y-1)$}
		\item $x^5y^2-2x^2y$\solution{$=x^2y(x^3y-2)$}
		\item $9xy^2-2x^3y$\solution{$=xy(9y-2x^2)$}
	\end{enumerate}
\end{multicols}

\section{Factoring by grouping}\label{sec:facbygroup}

We will now consider examples such as
\[
	x(x+7)+10(x+7)
\]
There are many approaches to factoring expressions like these, but we will generally keep the following guidelines in mind.
Put parentheses around factors with at least one common factor, usually pairs:
\begin{enumerate}
	\item Factor out the common factor from the first pair and rewrite
	\item Factor out the common factor from the next pair and rewrite
	\item At this stage we hope to see the same factors left in both parentheses- if not, then
	we might try to rearrange the original terms.
	\item The final step is to factor out the parenthetical terms (we will demonstrate these in our
	first example), leaving the other parenthetical term.
\end{enumerate} 

\begin{myexample}
Factor
\[
	x(x+7)+10(x+7)
\]
{}
\end{myexample}
\begin{myProof}
	In this case, the problem is at step 4. Do you see that the \gls{binomial} in each parentheses is the same? 
	So this gets `factored' out $(x+7)$ leaving $x()+10()$.
			
	Therefore the other binomial is $x+10$. The final answer is 
	\[
		(x+7)(x+10)
	\]
	or
	\[
		(x+10)(x+7)
	\]
	This can be checked by multiplying the two binomials using the method you prefer from
	the previous modules. 
\end{myProof} 

\begin{myexample}
Factor
\[
	x(y+9)-11(y+9)
\]
{}
\end{myexample}
\begin{myProof}
	Again this is at step 4, but with a different twist, and we will follow the same procedure as in 
	the previous example. A good first question is
	\begin{tightcenter}{\em Are the terms in the parentheses the same?}\end{tightcenter}
	The answer to this question is `yes', so we begin by factoring out $y+9$ from both terms leaving
	\[
		x()-11()
	\] 
	This makes the next factor $(x-11)$, and the answer is
	\[
		(y+9)(x-11)
	\]
	or
	\[
		(x-11)(y+9)
	\]
	because of the commutative property.
			
	Note that this can be checked by FOILing this expression, and making sure that it agrees with the original. Note that this can be checked by FOILing this expression, and making sure that it agrees with the original.
\end{myProof} 

\begin{myexample}
Factor
\[
	x^2+3x+5x+15
\]
\end{myexample}
\begin{myProof}
	This is the first example we have encountered where we need to do some manipulation before we can recognize that
	this can be broken into two factored terms. A natural question is: how do we decide which two terms to group together? A good first attempt (and it may often take more than one!)
	is to group the first two, and the second two. This therefore gives
	\[
		(x^2+3x)+(5x+15)
	\]
	Note that this would need considerably more care if the third `+' sign (between the $3x$ and the $5x$) was a `-'
	sign. 
			
	From here, we
	\begin{itemize}
		\item Factor the first group $x(x+3)$
		\item Factor the second group $5(x+3)$
	\end{itemize} 
	The terms in the parentheses are the same, so we can rewrite our expression as
	\[
		x(x+3)+5(x+3)
	\]
	from which we factor out the $(x+3)$ leaving $x()+5()$. So now the other binomial is
	$(x+5)$. The final answer is
	\[
		(x+3)(x+5)
	\]
\end{myProof} 

\begin{myexample}
Factor
\[
	x^3-3x^2+4x-12
\]
{}
\end{myexample}
\begin{myProof}
	We begin by inserting parentheses around the first 2 terms, and the last 2 terms (note that this may
	not always work- we might need to try another combination)
	\[
		(x^3-3x^2)+(4x-12)
	\]
	From here we
	\begin{itemize}
		\item Factor the first term: $x^2(x-3)$
		\item Factor the second term $4(x-3)$
	\end{itemize} 
	This terms in the parentheses are the same, so we can rewrite our expression as
	\[
		x^2(x-3)+4(x-3)
	\]
	As before, the parenthetical groups are the same, so we can factor out the $(x-3)$ term, 
	and our final answer is
	\[
		(x-3)(x^2+4)
	\]
\end{myProof} 

\begin{myexample}
Factor
\[
	x^3+6x^2-2x-12
\]
{}
\end{myexample}
\begin{myProof}
	The first differences we notice in this example as compared to those previous, are the
	`-' signs in front of the $-2x$ and $-12$. We can still proceed as before, but we must
	be careful to account for these. Before we start the factorization process, let us review 
	distributing a negative number.
			
	Consider
	\[
		-2(x+6)
	\]
	We distribute the $-2$ through the parenthesis and obtain
	\[
		-2x-12
	\]
	We can use this result with the expression in the above. We remove a factor of $x^2$ from the
	first two terms, and a factor of $-2$ from the second $2$ to give
	\[
		x^2(x+6)-2(x+6)
	\]
	Now we can factor as before, and our final result is
	\[
		(x^2-2)(x+6)
	\]
\end{myProof} 

\section{Factoring trinomials when $a=1$}
\textref{7.2}{422}%
In the previous section we saw how we could factor certain polynomials by grouping. In this section, we look at
trinomials that have a leading \gls{coefficient} of $1$. Remember that in our previous modules we used the \gls{FOIL} method
to expand binomial products. We considered examples such as
\begin{align*}
	(x+1)(x+3) & =		x^2+3x+x+3 \\	
	           & =		x^2+4x+3   
\end{align*}
The goal of this section is to develop the skills to go from the right hand side in the above (a \gls{trinomial}), to the left hand side-
in other words, to factor the trinomial.

There are a number of important observations that will help us to factor trinomials. We 
begin with a review of FOILing. 

\subsection{FOILing review}
We begin with a review of multiplying binomials using the FOIL method. Once we review
how the trinomial was put together it may be easier to see how to take it apart
\begin{center}
	\begin{tabular}{cccrrSc}
		\toprule
        Example & Binomials    & F     & O     & I     & {L}     & Trinomial   \\
		\midrule
		A       & $(x+2)(x+5)$ & $x^2$ & $5x$  & $2x$  & 10  & $x^2+7x+10$ \\
		B       & $(a-3)(a-2)$ & $a^2$ & $-2a$ & $-3a$ & 6   & $x^2-5a+6$  \\
		C       & $(c+4)(c-6)$ & $c^2$ & $-6c$ & $4c$  & -24 & $c^2-2c-24$ \\
		D       & $(m-3)(m+4)$ & $m^2$ & $4m$  & $-3m$ & -12 & $m^2+m-1$   \\
		\bottomrule
	\end{tabular}
\end{center}

Notice that the product of the first terms always yielded a squared term. Notice that the last
term of the trinomial is always the product of the two number. The middle term is found
by adding the O and I columns. 

Another helpful hint is to look at the signs of the terms for the trinomial compared to the signs
for the binomials. When the signs (connectors) between the terms is the same (as in examples A and B), 
the middle terms of the trinomial matches this same sign. In fact adding the middle terms always gives
the numerical value for the middle.

When the signs (connectors) between the terms are different, as in examples C and D, the middle
term could end up either sign. Looking at example C, we see that the middle term is negative
because the larges in absolute value $6$ was negative. The numerical value is the
difference of the two numbers. In example D, we see that the middle term was positive because
$4>3$, and 4 is a positive number. Using these sign rules will help you eliminate
many of the guesses you will need to make.

\begin{myDefinition} 
	\begin{itemize}
		\item if the numbers are the same sign then the middle term is the sum, and same sign as the last term
		\item if the numbers are different signs, then the middle term is the difference, and the sign is the same
		largest absolute value
	\end{itemize} 
\end{myDefinition}

\subsection{Products and sums of numbers}
We begin with an exercise in sums and products.  Consider the following table
\begin{center}
	\begin{tabular}{cSS}
		\toprule
        Two numbers & {Their product} & {Their sum} \\
		\midrule
		            & 8           & 6       \\
		            & 8           & 9       \\
		            & 8           & -9      \\
		\bottomrule
	\end{tabular}
\end{center}

In the left hand column, we wish to find two numbers that when multiplied together, give the middle column, and
when added together, give the third column. So we can fill in the first entry by noting that 
\[
	4+2=6,	
\]
and 
\[
	4\cdot 2=8
\]
So the numbers that we need to put into the first column are 
\begin{center}
	\begin{tabular}{cSS}
		\toprule
        Two numbers        & {Their product} & {Their sum} \\
		\midrule
		{\color{blue}4, 2} & 8           & 6       \\
		                   & 8           & 9       \\
		                   & 8           & -9      \\
		\bottomrule
	\end{tabular}
\end{center}



Note: choosing $4$ and $2$ was a very good first guess. In fact there are other factors of $8$ such as
$8$ and $1$, but these clearly give $8+1=9$, which is not $6$ (the required sum).

Similarly, in the second row we want two numbers that when multiplied together give $8$, and when 
added together give 9. We see that the required numbers are $8$ and $1$, since
\[
	8+1=9, \qquad 8*1=8
\]
We can therefore fill in the next row of our table
\begin{center}
	\begin{tabular}{cSS}
		\toprule
        Two numbers        & {Their product} & {Their sum} \\
		\midrule
		{\color{blue}4, 2} & 8           & 6       \\
		{\color{blue}8, 1} & 8           & 9       \\
		                   & 8           & -9      \\
		\bottomrule
	\end{tabular}
\end{center}

The last row is interesting, as the product is {\em negative}, and yet their product is still {\em positive}. Remember
from elementary arithmetic, that a negative number times a negative number is a positive. For example
\[
	(-2)(-5)=10.
\]
We use this result in the last row of the table. The factors of 8 are 
\[
	1, 8, \qquad 2,4,	\qquad -2, -4, \qquad -8, -1.
\]

The only pair that adds up to -9 is -8 and -1. So we complete the final row of the table with
\begin{center}
	\begin{tabular}{cSS}
		\toprule
        Two numbers           & {Their product} & {Their sum} \\
		\midrule
		{\color{blue}4, 2}    & 8           & 6       \\
		{\color{blue}8, 1}    & 8           & 9       \\
		{\color{blue}-8, -1 } & 8           & -9      \\
		\bottomrule
	\end{tabular}
\end{center}

{\bfseries\itshape Before reading the rest of this section, complete the following table. Answers
	are shown in the footnote \footnote{5,4; 10,2; -10,-2; 7,2; -13,-1; -3, 1; -7,1; }}
\begin{center}
	\begin{tabular}{cSS}
		\toprule
		Two numbers & {Their product} & {Their sum} \\
		\midrule
		            & 20              & 9           \\
		            & 20              & 12          \\	
		            & 20              & -12         \\	
		            & 14              & 9           \\
		            & 13              & -14         \\	
		            & -3              & -2          \\	
		            & -7              & -6          \\	
		            & -7              & 6           \\	
		            & 60              & -23         \\	
		            & -15             & -2          \\	
		            & -17             & 16          \\	
		\bottomrule
	\end{tabular}
\end{center}

\begin{myexample}
Factor
\[
	x^2+7x+10
\]
{}
\end{myexample}
\begin{myProof}
	\begin{itemize}
		\item First we determine the signs on the connectors of each binomial. Since
		the middle term and the last term are positive, we will have
		\[
			(\phantom{x}+\phantom{x})(\phantom{x}+\phantom{x})
		\]
		\item Next we determine where the first term (product) could have come from. 
		The choices are only $x$ and $x$. 
		\[
			(x+\phantom{x})(x+\phantom{x})
		\]
		\item To determine the numbers, we need to look at the last term. Since this will
		be the product of the factors, we need to write the number as the product of possible
		factors
		\[
			10 = 10(1)	\qquad 10 = 5 \cdot 2	
		\]
		\item Since the middle term is $7$ we know that we need the sum of the factors to be 7. 
		So we therefore try
		\[
			(x+5)(x+2)
		\]
		\item We should check our answer using the FOIL technique (exercise).
		\item Because of the commutative property of multiplication, the answer can also be written as
		\[
			(x+2)(x+5)
		\]
	\end{itemize}
\end{myProof} 

\begin{myexample}
Factor
\[
	x^2-14x+45
\]
{}
\end{myexample}
\begin{myProof}
	\begin{itemize}
		\item First we determine the signs on the connectors of each binomial. Since
		the middle term and the last term are negative, we will have
		\[
			(\phantom{x}-\phantom{x})(\phantom{x}-\phantom{x})
		\]
		\item Next we determine where the first term (product) could have come from. 
		The choices are only $x$ and $x$. 
		\[
			(x-\phantom{x})(x-\phantom{x})
		\]
		\item To determine the numbers, we need to look at the last term. Since this will
		be the product of the factors, we need to write the number as the product of possible
		factors
		\[
			45=45(1)\qquad 45=3(15)\qquad 5(9)
		\]
		\item Since the middle term is $-14$ we know that we need the sum of the factors to be -14. 
		So we therefore try
		\[
			(x-5)(x-9)
		\]
		\item We should check our answer using the FOIL technique (exercise).
		\item Because of the commutative property of multiplication, the answer can also be written as
		\[
			(x-9)(x-5)
		\]
	\end{itemize}
\end{myProof} 

\begin{myexample}\label{ex:factoring1}
Factor
\[
	y^2+5y-24
\]
{}
\end{myexample}
\begin{myProof}
	\begin{itemize}
		\item First we determine the signs on the connectors of each binomial. Since
		the middle term is positive, and the last term is negative, we will have
		\[
			(\phantom{x}-\phantom{x})(\phantom{x}+\phantom{x})
		\]
		\item Next we determine where the first term (product) could have come from. 
		The choices are only $y$ and $y$. 
		\[
			(y-\phantom{y})(y+\phantom{y})
		\]
		\item To determine the numbers, we need to look at the last term. Since this will
		be the product of the factors, we need to write the number as the product of possible
		factors
		\begin{tightcenter}
			$-24 = -1(24)\qquad -24=-24(1) \qquad -24=-12(2) \qquad -24=-2(12)$ \\
			$-24 = -8(3) \qquad -24=-3(8)\qquad -24=-6(4) \qquad -24 = -4(6)$
		\end{tightcenter}
		\item Since the middle term is $5$ we know that we need the sum of the factors to be $5$.
		So we therefore try
		\[
			(y-3)(y+8)
		\]
		\item We should check our answer using the FOIL technique (exercise).
		\item Because of the commutative property of multiplication, the answer can also be written as
		\[
			(y+8)(y-3)
		\]
	\end{itemize}
	Note: if we had tried
	\[
		(y+3)(y-8)
	\]
	then this would not have worked, as the O and I terms from FOIL do not add together to give $5$.
\end{myProof} 

\begin{myexample}
Factor
\[
	w^2+12w-64
\]
{}
\end{myexample}
\begin{myProof}
	\begin{itemize}
		\item First we determine the signs on the connectors of each binomial. Since
		the middle term is positive, and the last term is negative, we will have
		\[
			(\phantom{x}-\phantom{x})(\phantom{x}+\phantom{x})
		\]
		\item Next we determine where the first term (product) could have come from. 
		The choices are only $x$ and $x$. 
		\[
			(x-\phantom{x})(x+\phantom{x})
		\]
		\item To determine the numbers, we need to look at the last term. Since this will
		be the product of the factors, we need to write the number as the product of possible
		factors
		\begin{tightcenter}
			$-64=-64(1)\qquad -64=-1(64)\qquad -64=-32(2) \qquad -64 = -2(32) $\\
			$-64=-16(4)\qquad -64 = -4(16)\qquad -64 = -8(8)$
		\end{tightcenter}
		\item Since the middle term is $12$ we know that we need the sum of the factors to be $12$.
		So we therefore try
		\[
			(x-4)(x+16)
		\]
		\item We should check our answer using the FOIL technique (exercise).
		\item Because of the commutative property of multiplication, the answer can also be written as
		\[
			(x+16)(x-4)
		\]
	\end{itemize}
	Note: in this example there were quite a few different options to choose from. This is typical, and you 
	should expect to do some trial error for each exercise. Experience and practise will help to guide you.
\end{myProof}

\begin{myexample}
Factor
\[
	2x^2+10x-48
\]
{}
\end{myexample}
\begin{myProof}
	At first glance, it may appear that this example should not be in this section, as $a\ne 1$; however, we note
	that as a first step we may remove a factor of 2 from each term
	\[
		2x^2+10x-48 = 2(x^2+5x-24)
	\] 
	We recognize the expression within the () as \cref{ex:factoring1} (except with $x$ instead of $y$),
	so we can complete this example as follows
	\[
		2(x^2+5x-24)	=	2(x+8)(x-3)	
	\]
	As in the previous examples, we should check our answer by FOILing (exercise).
\end{myProof} 

\begin{myexample}
Factor the following.
\drillandskill
\end{myexample}
\begin{multicols}{2}
	\begin{enumerate}
		\item $x^2 + 4x+3 = \solution{(x+{3})}\solution{(x+{1})}$
		\item $x^2+5x+4 = \solution{(x+{4})}\solution{(x+{1})}$
		\item $x^2-7x+12=\solution{(x{-4})}\solution{(x{-3})}$
		\item $x^2+16x-17=\solution{(x{+17})}\solution{(x{-1})}$
		\item $x^2+13x+30=\solution{(x{+10})}\solution{(x{+3})}$
		\item $x^2-13x+30=\solution{(x{-10})}\solution{(x{-3})}$
		\item $x^2+12x+27=\solution{(x{+9})}\solution{(x{+3})}$
		\item $x^2+6x-27=\solution{(x{+9})}\solution{(x{-3})}$
		\item $x^2-6x-27=\solution{(x{-9})}\solution{(x{+3})}$
		\item $x^2+10x+21=\solution{(x{+7})}\solution{(x{+3})}$
		\item $x^2+13x+22=\solution{(x{+11})}\solution{(x{+2})}$
		\item $x^2-24x+80=\solution{(x{-20})}\solution{(x{-4})}$
		\item $x^2+18x+81=\solution{(x{+9})}\solution{(x{+9})}$
		\item $x^2+29x+100=\solution{(x{+25})}\solution{(x{+4})}$
	\end{enumerate} 
\end{multicols}

\begin{myexample}
Factor the following more advanced problems
\end{myexample}
\begin{multicols}{2}
	\begin{enumerate}
		\item $
		\begin{aligned}[t]
			2x^2-4x-30 & =  \solution{2({x^2-2x-15})}  \\
			           & =  \solution{2(x{-5})(x{+3})} 
		\end{aligned}
		$
		\item $w^4+3w^2+2 = \solution{(w^2+2)(w^2+1)}$
		\item $
		\begin{aligned}[t]
			3w^4+9w^2+6 & = \solution{3(w^4+3w^2+2)}    \\
			            & =  \solution{3(w^2+2)(w^2+1)} 
		\end{aligned}
		$
		\item $
		\begin{aligned}[t]
			4x^2+8x+4 & =  \solution{4(x^2+2x+1)} \\
			          & =  \solution{4(x+1)(x+1)} 
		\end{aligned}
		$
		\item $
		\begin{aligned}[t]
			-x^2-3x-2 & =  \solution{-(x^2+3x+2)} \\
			          & =  \solution{-(x+2)(x+1)} 
		\end{aligned}
		$
		\item $
		\begin{aligned}[t]
			-4x^2-20x-16 & =  \solution{-4(x^2+5x+4)} \\
			             & =  \solution{-4(x+4)(x+1)} 
		\end{aligned}
		$
		\item $
		\begin{aligned}[t]
			x^3+3x^2+2x & =  \solution{x(x^2+3x+2)} \\
			            & =  \solution{x(x+2)(x+1)} 
		\end{aligned}
		$
		\item $
		\begin{aligned}[t]
			x^4+3x^3+2x^2 & =  \solution{x^2(x^2+3x+2)} \\
			              & =  \solution{x^2(x+2)(x+1)} 
		\end{aligned}
		$
		\item $x^2+5xy+4y^2=\solution{(x+4y)(x+y)}$
		\item $x^2-5xy+4y^2=\solution{(x-4y)(x-y)}$
		\item $a^2+8ab+7b^2=\solution{(a+7b)(a+b))}$
		\item $x^2+9xy+14y^2=\solution{(x+7y)(x+2y)}$
	\end{enumerate}
\end{multicols}

\section{Factoring trinomials when $a\ne 1$}
\textref{7.3}{430}%
We have so far looked at two forms of factoring
\begin{itemize}
	\item factoring by grouping
	\item factoring by reversing the FOIL procedure
\end{itemize} 
In all of our examples so far, the leading coefficient of the polynomial has been $a=1$,
for example
\[
	x^2+4x+3
\]
In this section we will study a slightly more complicated set of polynomials that have leading
coefficient {\em other than} 1. We will restrict ourselves to trinomials of \gls{degree} 2 (which we will
later term {\em Quadratics}), which have the form
\[
	ax^2+bx+c
\]
We will attempt to factor these using the `$ac$ method', which relates to using the $a$ and the $c$
in the above. This method is particularly effective when $a\ne 1$.

Note that factoring trinomials when $a\ne 1$ requires patience, so do not be disappointed if it
takes some time to sharpen the skill.

\subsection{The $ac$ method}
\begin{myexample}
Factor
\[
	8x^2-14x-15
\]
{}
\end{myexample}
\begin{myProof}
	The first observation we make about this trinomial is that the leading term has a coefficient
	of $8$, which significantly hinders our previous approach to factoring.
				
	We will organize our information in the following table
	\begin{itemize}
		\item the first column displays $a$, the second displays $c$, and the third displays $ac$
		\item the fourth column displays every combination of (integer) factors of the product $ac$
		\item we sum the factors in the fifth column, and compare them to the middle term in the
		original trinomial
	\end{itemize} 
				
	\begin{list}{}{%
			\setlength{\leftmargin}{-1.5cm}
			\setlength{\rightmargin}{1.5cm}}%
		\item[]%
		\begin{tabular}{SSScSSp{4.5cm}}
			\toprule
			&		&		&			&	\multicolumn{2}{c}{Need these 2 columns to match}	&	\\
			\midrule
			{$a$} & {$c$} & {$ac$} & Factors & {Sum of factors} & {middle term} & Match?                                          \\
			8     & -15   & -120   & -1,120  & 119              & -14           & No. Also note that the middle term is negative, 
			so when we list the factors let's make the factor
			with the largest absolute value negative so the sum will be negative\\
			8     & -15   & -120   & 1,120   & -119             & -14           & No- but at least it's the same sign!            \\
			8     & -15   & -120   & 2,-60   & -58              & -14           & No- but getting closer!                         \\
			8     & -15   & -120   & 3,-40   & -37              & -14           & No- but getting closer!                         \\
			8     & -15   & -120   & 4,-30   & -26              & -14           & No- but getting closer!                         \\	
			8     & -15   & -120   & 5,-24   & -19              & -14           & No- but getting closer!                         \\
			8     & -15   & -120   & 6,-20   & -14              & -14           & Bingo!                                          \\
			\bottomrule
		\end{tabular}
	\end{list} 
				
	We still have some work to do, as the factors 6 and -20 will not appear in our final answer. 
				
	Recall that in all the binomial multiplication that we did, the middle term came from adding the Inner and the Outer terms. We now need
	to find those 2 terms, which is where the 6 and -20 come in. Notice that
	\[
		6x-20x = -14x
	\]
	which is the middle term of our trinomial. 
				
	This means that we can rewrite our trinomial as
	\[
		8x^2 - 14x-15 = 8x^2 -20x+6x-15
	\]
	We can now use the skills we learnt in factoring by grouping (see \vref{sec:facbygroup}) and proceed as follows
	\begin{align*}
		8x^2 - 14x-15 & =		8x^2-20x+6x-15                             \\
		              & =		(8x^2-20x)+(6x-15)                         \\
		              & =		4x{\color{red}(2x-5)}+3{\color{red}(2x-5)} \\
		              & =		(4x+3){\color{red}(2x-5)}                  
	\end{align*} 
	What if we had written the middle term as $6x-20x$ instead? We would still end up at the same answer, but the
	working would be as follows
	\begin{align*}
		8x^2-14x-15 & =		8x^2+6x-20x-15                               \\
		            & =		(8x^2+6x)+(-20x-15)                          \\
		            & =		2x{\color{blue}(4x+3)}-5{\color{blue}(4x+3)} \\
		            & =		(2x-5){\color{blue}(4x+3)}{}                 
	\end{align*} 
	which is obviously the same as before.
\end{myProof} 

\begin{myexample}
Factor
\[
	4w^2-24w-64
\]
{}
\end{myexample}
\begin{myProof}
	At first glance this seems like a trinomial with $a\ne 1$, but if we examine each term individually we notice
	that each of the coefficients is divisible by 4. We can therefore factor out 4 from each term and rewrite the 
	trinomial as
	\[
		4w^2-24w-64 = 4(w^2-6w-16)
	\]
	Now we need to see if we can factor $w^2-6w-16$, which is straight forward since $a=1$. 
	\begin{center}
		\begin{tabular}{SSScSSl}
			\toprule
			&		&		&			&	\multicolumn{2}{c}{Need these 2 columns to match}	&	\\
			\midrule
			{$a$} & {$c$} & {$ac$} & {Factors} & {Sum of factors} & {middle term} & Match? \\
			1     & -16   & -16    & 1,-16     & -16              & -6            & No     \\
			1     & -16   & -16    & -2,8      & 6                & -6            & No     \\
			1     & -16   & -16    & -4,4      & 0                & -6            & No     \\
			1     & -16   & -16    & 2,-8      & -6               & -6            & Yes    \\
			\bottomrule
		\end{tabular}
	\end{center} 
				
	We can use this table to write
	\[
		w^2-6w-16 = (w+2)(w-8)
	\]
	The answer to our original question is therefore
	\begin{align*}
		4w^2-24w-64 & =	 	4(w^2-6w-16) \\
		            & =		4(w-8)(w+2)   
	\end{align*} 
\end{myProof} 

\subsection{Trial and error `method'}
This can not really be described as a method, but another way to factor is by trial and error. Consider	again factoring
\[
	8x^2-14x-15
\]
We could try any one of the following factorizations until we achieve the correct result by FOILing
\begin{align*}
	8x^2-14x-15 & \stackrel{?}{=}		(8x-3)(x+5)  \\
	8x^2-14x-15 & \stackrel{?}{=}		(8x+3)(x-5)  \\
	8x^2-14x-15 & \stackrel{?}{=}		(8x-1)(x+15) \\
	8x^2-14x-15 & \stackrel{?}{=}		(8x-15)(x+1) \\
	8x^2-14x-15 & \stackrel{?}{=}		(4x-3)(2x+5) \\
	8x^2-14x-15 & =			(4x+3)(2x-5)              \\
\end{align*} 

\begin{myexample}
\drillandskill
Factor the following - notice that both connectors are $+$:
\begin{multicols}{2}
	\begin{enumerate}
		\item $2x^2+3x+1$\solution{$=(2x+1)(x+1)$}
		\item $2x^2+5x+2$\solution{$=(2x+1)(x+2)$}
		\item $2x^2+7x+6$\solution{$=(2x+3)(x+2)$}
		\item $2x^2+11x+12$\solution{$=(2x+3)(x+4)$}
	\end{enumerate}
\end{multicols}

Factor the following -notice that both connectors are $-$:
\begin{multicols}{2}
	\begin{enumerate}
		\item $2x^2-3x+1$\solution{$=(2x-1)(x-1)$}
		\item $2x^2-5x+2$\solution{$=(2x-1)(x-2)$}
		\item $2x^2-7x+6$\solution{$=(2x-3)(x-2)$}
		\item $2x^2-11x+12$\solution{$=(2x-3)(x-4)$}
	\end{enumerate}
\end{multicols}

Factor the following - notice that the connectors are one of each:
\begin{multicols}{2}
	\begin{enumerate}
		\item $3x^2-2x-1$\solution{$=(3x+1)(x-1)$}
		\item $3x^2-5x-2$\solution{$=(3x+1)(x-2)$}
		\item $3x^2-4x-4$\solution{$=(3x+2)(x-2)$}
		\item $3x^2-2x-8$\solution{$=(3x+4)(x-2)$}
	\end{enumerate}
\end{multicols}

Factor the following (miscellaneous)
\begin{multicols}{2}
	\begin{enumerate}
		\item $6x^2+5x+1$\solution{$=(3x+1)(2x+1)$}
		\item $6x^2-x-1$\solution{$=(3x+1)(2x-1)$}
		\item $10x^2+9x+2$\solution{$=(5x+2)(2x+1)$}
		\item $10x^2+x-2$\solution{$=(5x-2)(2x+1)$}
		\item $12x^2+41x+24$\solution{$=(4x+3)(3x+8)$}
	\end{enumerate}
\end{multicols}

Factor the following- look for a common factor first!
\begin{enumerate}
	\item $6x^4+13x^3+6x^2$\solution{$=x^2(3x+2)(2x+3)$}
	\item $6x^4+5x^3-6x^2$\solution{$=x^2(3x-2)(2x+3)$}
	\item $8x^5y+22x^4y+15x^3y$\solution{$=x^3y(4x+5)(2x+3)$}
	\item $16x^5y-60x^4y+50x^3y$\solution{$=2x^3y(4x-5)(2x-5)$}
\end{enumerate}

Factor the following:
\begin{multicols}{2}
	\begin{enumerate}
		\item $4x^2+4xy+y^2$\solution{$=(2x+y)(2x+y)$}
		\item $6x^2+5xy+y^2$\solution{$=(2x+y)(3x+y)$}
	\end{enumerate}
\end{multicols}
\end{myexample}

